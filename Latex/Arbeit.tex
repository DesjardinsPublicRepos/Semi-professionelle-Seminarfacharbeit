\documentclass[a4paper, 12pt]{scrartcl}

\usepackage[utf8]{inputenc}
\usepackage[english, ngerman]{babel}
\usepackage[T1]{fontenc}
\usepackage{amsmath}
%für Bilder
\usepackage{graphicx}
%deutsche anführungszeichen
\usepackage{csquotes}\MakeOuterQuote{"}
%ränder
\usepackage[left=3cm, right=4cm, top=3cm, bottom=2.5cm]{geometry}
\usepackage[numbers, round]{natbib}
%spacing
\usepackage[onehalfspacing]{setspace}
%Abkürzungsverzeichnis
\usepackage[printonlyused, footnote]{acronym}
%Anzeigen der restlichen Verzeichnisse
\usepackage{tocbibind}
%Literaturverzeichnis
\bibliographystyle{alphadin}

\newcommand*{\bildquelle}{%
  \footnotesize Quelle:
}
\hyphenation{Schleu-sing-en Haupt-ein-kaufs-mö-glich-keit Nie-drig-preis-po-li-tik}

\title{Seminarfacharbeit}
\author{Toni Hausdörfer, Fabian Beez}
\date{27. Mai 2020}

\begin{document}
    \begin{titlepage}

    \raggedright
        Hennebergisches Gymnasium Georg Ernst\\
        Klosterstraße 2-4\\
        98554 Schleusingen
        
    \raggedleft\vspace*{-1.9cm}
        Jahrgang 2020/21
        
    
    \vfill\vfill\vfill\vfill\vfill\vfill

    \centering
        \LARGE\textbf{Seminararbeit}
        \vfill
        \large Onlinehandel und dessen Einfluss auf Kleinstädte und Dörfer im ländlichen Raum?\\[\baselineskip]\vfill
        
        von\\
        Toni Hausdörfer und Fabian Beez
    

    \vfill\vfill\vfill\vfill

    \raggedright
    Betreuende Lehrkraft:   Frau[?] Richter\\
    Abgabetermin:\\[\baselineskip]
    
\begin{tabular}[h]{|l|l|l|l|l|l|}
    \hline
    Bewertung & Note & Note in Worten & Punkte &  & Punkte \\
    \hline
    schriftliche Arbeit & & & & x 3 & \\
    \hline
    Präsentation & & & & x 1 & \\
    \hline
\end{tabular}

\vfill

\raggedleft
    Summe: \rule{1.5cm}{.4pt}\\
    Gesamtleistung nach \$ 61(7) GSO = Summe :2 (gerundet): \rule{1.5cm}{.4pt}\\[\baselineskip]\vfill\vfill\vfill
    \begin{tabular}{@{}l@{}}\hline
        Unterschrift des Kursleiters
    \end{tabular}
    
\vfill\vfill
\end{titlepage}

\newpage

    \vspace*{5cm}
\section*{Erklärung der Urheberschaft}
Hiermit versichere ich, dass ich die vorliegende Arbeit
ohne Hilfe Dritter und ohne Benutzung anderer als der angegebenen
Hilfsmittel angefertigt habe. Alle aus fremden Quellen direkt oder
indirekt übernommenen Gedanken sind als solche kenntlich gemacht. Diese
Arbeit wurde bisher in gleicher oder ähnlicher Form in keiner anderen
Prüfungsbehörde vorgelegt oder veröffentlicht.

\vfill

\noindent\dotfill\\
\hspace{2cm} Ort, Datum \hfill Unterschrift \hspace{2cm}

\vfill\vfill\vfill\vfill\vfill
\newpage


    \tableofcontents 
        \newpage

    \section{Einleitung}
        %?: mode beetz, ..., in schleusingen schließen immer mehr läden und es werden kaum neue eröffent(abgesehen von restoraunnts). gleichzeitig wird der onlineeinkauf immer attraktiver


%Mode Beetz, ... - in Schleusingen schließen immer mehr Läden, währen kaum neue eröffnet werden. Gleichzeitig wird der Onlineeinkauf immer attraktiver. Da kommt schnell die Befürchtung auf, dass die Innenstadt verschwinden wird. 


%tühringer Landkreis dafür bekannt, dass er technisch hinterherhängt -> wie sieht das in anderen bundesländern aus und was können wir daraus lernen?
Onlinehandel spielt eine immer größer werdende Rolle im Leben von Menschen fast aller Altersschichten. Aufgrund des allmählichen Verschwindens von Geschäften im Gebiet um Schleusingen haben wir uns die Frage gestellt, welche Rolle der Verkauf von Waren über das Internet in unserem Landkreis spielt, und ob die Schließungen einiger Geschäfte auf den Rückgang der Nachfrage für lokale Einzelhändler zurückzuführen ist.

Im Rahmen unserer Arbeit werden wir das Zutreffen folgender Thesen einschätzen:
\begin{itemize}
    \item Der Onlinehandel führt zu einem allmählichen Aussterben von stationären Händlern im ländlichen Bereich.
    
    \item Es gibt Waren, die nicht/kaum online gekauft werden.

    \item Der Onlinehandel schafft weniger Arbeitsplätze als indirekt verringert werden.
\end{itemize}

Außerdem werden wir auf Basis unserer Ergebnisse ein Konzept entwickeln, wie die Nachfrage und der Verkauf von Waren in unserem Landkreis erhöht werden kann.

        \newpage
    
    \section{Historie}
        \subsection{Entstehung}
        \subsection{Entwicklung mit Blick auf Großfirmen wie PayPal}
        \subsection{Beeinflussung der Sozialstrukturen}
        \newpage
        
    \section{PayPal}
        \subsection{Entstehung und grobe Funktionsweise}
        \subsection{Entwicklung}
        \subsection{Einfluss auf die Gesellschaft}
        \newpage
        
    \section{Konsumverhalten}
            Die Zusammensetzung der Güternachfrage hat sich in den letzten Jahrzehnten stark verändert. Insbesondere unter dem Aspekt des Onlinehandels ist es nun für Firmen wichtig, ihre Verkaufskonzepte evt. zu erweitern oder zu aktualisieren. Im folgenden Teil werde ich die Entwicklung sowie das aktuelle Kaufverhalten von Konsumenten mithilfe einer [Umfrage] analysieren und auf Basis der Ergebnisse möglich Folgen formulieren.

        \subsection{Veränderungen im Konsumverhalten über die Zeit}
            \iffalse
https://edoc.sub.uni-hamburg.de/hcu/volltexte/2017/370/pdf/Ebert_Kirsten.pdf Anfang
\fi
%synonym handelsmodelle

\begin{folding} % Einleitung

Im folgendem werde ich Verkäufer aller Art ähnlich wie im Buch "Das E-Commerce Buch: Marktanalysen - Geschäftsmodelle - Strategien" unterteilen: in Online-Marktplätze, Online-Händler, Kataloversender, stationäre Händler und Hersteller\cite[S. 15ff]{Graf}. Dabei sind Online-Marktplätze eine Art Online-Vermittler zwischen Kunden und Verkäufer, Online-Händler bieten dagegen nur eigene, meist sehr spezialisierten Sortimente an. Kataloversender verhalten sich ähnlich: sie versenden ihr Sortiment direkt an Kunden. Stationäre Händler verkaufen im Gegensatz zu den genannten Vertreibsstrukturen in Fillialen und sind mit Kataloghändlern am stärksten von den Änderungen der letzten Jahrzenten betroffen. Während sich die genannten Unternehmensarten meistens am Ende der Verkaufskette befinden, stehen Hersteller am Anfang: sie stellen Güter her und sind dementsprechend, insofern sie nicht selber verkaufen, auf weitere Unternehmen für Verkauf und Vermarktung angewiesen(ebd.). %vor und -nachteile?; beispielsunternehmen

\end{folding}

\begin{folding} % KATALOGHÄNDLER

Kataloghändler sind die größten Verlierer der letzten Jahrzehnte: mit der Entwicklung des Onlinehandels ist ab 2002 schon ein Rückgang der Nachfrage zu spüren – einige eröffnen eigene Online-Shops\cite[S. 24f]{Graf}, jedoch oft mit wenig Erfolg\cite[S. 38]{Graf}. 2015 sind Katalogversender fast ausschließlich verschwunden oder zu Online- und Einzelhandel konvertiert, da sie kaum einen Mehrwert im Vergleich zum klassisschen Onlinehandel bieten\cite[S. 47]{Graf}. So prognostiziert beispielsweise 2012 IFH Retail Consultants einen einen sinkenden Anteil des Online-Umsatzes von 24.9\% zu 23.9\% in den folgenden 2 Jahren\cite[S. 20]{evilcom}. Tatsächlich fiel der Anteil aber ganze 4.6\% - knapp das fünffache des erwarteten Wertes\cite{statista-vertriebsformen}. %Zusätzlich wird ein Umsatzanteil von 16.6\% für 2019 vorrausgesagt. ::: zu viel wirtschaft

\end{folding}

\begin{folding} % STATIONÄR
    % nitt S 54 these, dass einzelhandel zurückkommt
Der stationäre Handel ist durch die steigende Relevanz des Onlinehandels auch weniger gefragt denn je und versucht mit strukturellen Änderungen dagegen anzukämpfen. Einige Einzelhändler eröffenen paralel zu ihrem Geschäft einen Online-Shop, andere bieten die Möglichkeit, Waren online in den Laden zu bestellen und diese dort anzuholen – sogenanntes "Multichannel-Marketing", das Ansprechen der Kunden über mehrere Wege\cite[S. 34f]{Graf}. Jedoch fahren die neuen Strkturen nur wenig Erfolge ein – so erhöhen sie zwar die Onlinepräsenz, bieten jedoch nur einen geringen Mehrwert im Vergleich zu den bekannten Onlineriesen wie Amazon\cite[S. 34f]{Graf}. 
Schließlich stellt sich die Frage, welche Vorteile der stationäre Handel noch bieten kann, um die im Vergleich zum Onlinehandel deutlich höheren Preise zu rechtfertigen - denn pure Onlinehändler haben deutlich dünnere Kostenstrukturen\cite[S. 14]{evilcom}. So müssen sie etwa keine Miete für Geschäfte zahlen und kommen mit deutlich weniger Angestellten aus, folglich weniger Kosten.

Einer dieser Vorteile ist in der Theorie der soziale Aspekt des Einkaufens, der laut Nitt-Drießelmann vorallem für über-50-Jährige, die über viel Freizeit verfügen, eine immer größer werdende Rolle spielen wird\cite[S. 43f]{Nitt}:
\begin{quote}
"Als Mittel gegen Vereinsamung und Anonymisierung im Alltag wird die soziale Komponente beim Einkaufen [...] zunehmend an Bedeutung gewinnen."\cite[S. 43]{Nitt}
\end{quote} 
So soll in Zukunft der Wunsch nach Begegnungen mit bekannten Personen und Beratung zunehmen und die Zusammensetzung von sozialen Kontakten eine immer wichtigere Rolle spielen(ebd.).
Außerdem müssen stationäre Händler stärker auf die geänderten Wünsche an Sie von Konsumenten eingehen. Beispielsweise wollen Sie einen bequemen Einkauf mit langen Öffnungszeiten, eine übersichtliche Warenpräsentation sowie eine möglichst große Produktauswahl auf einer so kleinen Verkaufsfläche wie möglich\cite[S. 61]{Nitt}.

Zu dem kommt, dass in Deutschland bedeutende demografische Änderungen bevorstehen: so schrumpft und altert die Gesamtbevölkerung, folglich muss der stationäre Handel sich auf einen zusätzlichen Nachfragerückgang einstellen sowie die Bedürfnisse von Senioren stärker beachten - insbesondere auf dem Land, denn in Metropolen soll das Durchschnittsalter nahezu konstant bleiben\cite[S. 32ff]{Nitt}. Diese Chance kann er aber nur nutzen, indem er stärker auf die Bedürfnisse der älteren Bevölkerung, die 2050 etwa 59\% der Kaufkraft ausmachen soll, eingeht\cite[S. 64]{Nitt}. So kaufen Sie oft Qualitätsprodukte in den Bereichen Gesundheit, Wohnen und Energie - jedoch kaum langlebige Konsumgüter, da Sie diese schon besitzen; zusätzlich sehen sie kein Problem damit, für kompetente Beratung mehr zu bezahlen\cite[S. 41f]{Nitt}. Folglich muss der stationäre Handel die Produktauswahl sowie Erreichbarkeit und Übersichtlichkeit auf die alternde Bevölkerung anpassen, um die beste Kaufoption für Sie zu bleiben\cite[S. 64]{Nitt}.

Obwohl in Zukunft die Bevölkerung Deutschlands schrumpfen wird, ist eine erhöhte Anzahl von Haushalten zu erwarten - durch eine "Zerstreuung" der Haushaltsstruktur. So soll es 2030 1.8 Mio. weniger Mehrpersonenhaushalte geben, dafür aber 1.4 Mio. Einpersonen- sowie 1.6 Mio. Zweipersonenhaushalte mehr als 2010\cite[S. 35]{Nitt}, was zu einer automatischen Erhöhung der Wochnfläche pro Person führt. So werden Haushaltsprodukt- und Möbelverkäufer in den nächsten Jahren weniger von Insolvenzen betroffen sein wie Unternehmen anderer Branchen.

Zusätzlich gibt es beim stationären Handel Produkte, die nicht oder nur schwer durch andere Vertriebswege abzudecken sind, wie etwa beratungsintensive Waren
%https://de.statista.com/statistik/daten/studie/201914/umfrage/einkaufsverhalten-im-onlinehandel-vs-einzelhandel-nach-produktgruppen/

%"feeling" des produktes

%handwerker: kaum durch online ersetzbar, allerdings ist ein pur stationärer handwerkerbetrieb auch nicht zukunftsfähig.

\end{folding}

\begin{folding} % ONLINE

Die Onlinehändler und -marktplätze sind die Gewinner der letzten 20 Jahre – die Verkäufswerte wuchsen ab der Jahrtausendwende konstant an und stellen in vielen Branchen für andere Vertriebsstrukturen eine ernst zu nehmende Konkurrenz dar\cite{wolf}. Vorerst wechseln Konsumenten von Katalogen, ab 2010 auch viele Nutzer anderer Verkaufswege, da das Kaufen online fast immer einen Preisvorteil bietet\cite[S. 31]{Graf}.
\begin{figure}[h]
    \begin{center}
        \includegraphics[width=8cm]{media/Fabian-konsumwandel.png}
        \caption{Kaufprozess im Vergleich – Stationär und E-Commerce}
        \label{konsumwandel}
        \bildquelle Björn Schäfers, Social Shopping für Mode, Wohnen und Lifestyle am Beipiel Smatch.com in Web-Exzellenz im E-Commerce, Gabler, S. 313 %lieber quelle e com buch???
    \end{center}
\end{figure} 
Außerdem hat sich unter Nutzern des Onlinehandels ein neues Konsumverhalten entwickelt. Bis dahin wa es üblich, zuerst den Anbieter, danach das zu kaufende Produkt auszuwählen; jedoch hat der Onlinehandel dieses Verhalten invertiert, da das Vergleichen mehrerer Produkte im Internet um ein vielfaches einfacher ist als stationär, allein schon aufgrund des Zeitaufwands für das Besuchen von mehrern Geschäften\cite[S 22f]{Graf}. Auch die Beratung des stationären Handels spielt hier eine Rolle: Online-Käufer informieren sich oft selber und können auf Basis ihrer Recherche das optimale Produkt wählen, während beim Kauf vor Ort meist auf einen bestimmten Verkäufer und dessen Beratung vertraut wird\cite[S. 15f]{evilcom}. Infolge dessen sind Verbraucher, die bereits einmal online eingekauft haben, oft sehr preissensibel - und das auch bei Käufen vor Ort\cite[S. 60]{Nitt}
Neu unter Konsumenten ist auch das Bedürfnis nach individiuellen und auf den Käufer angepassten Produkten\cite[S. 43]{Nitt}, was wahrscheinlich durch die extrem große Außwahl bei dem Online-Shopping ausgelöst wurde. In diesem Aspekt kann der stationäre Einzelhandel schlicht nicht mithalten, da Raum für Produkte stärker begrenzt und preisintensiv ist.

\end{folding}

\begin{folding} % HERSTELLER

Ähnlich wie der stationäre Handel sehen Hersteller den Onlinehandel zuerst in einem negativen Licht - aufgrund von untransparenten Verkäufern und möglichen negativen Imageeffekten\cite[S. 20]{Graf}. Außerdem entstehen 2002 erste, sogenannte "Powerseller", die im Großhandel Markenprodukte kaufen und deutlich unterhalb des Einzelhandels-Preisniveaus verkaufen\cite[S. 26]{Graf}. Mit der Zeit bauen Marken jedoch verzögert, aber schneller als der stationäre Handel immer mehr eigene Verkaufsportale, um ihre Güter direkt ohne eine Zwischeninstanz zu verkaufen und steigern ihren Umsatz damit bedeutend – beispielsweise verlässt Hugo Boss 2013 Zalando um Produkte über die eigene Website hugoboss.com zu verkaufen\cite[S. 48f]{Graf}. Außerdem wird durch das direkte Feedback eine bessere Produktentwicklung und Kundensupport ermöglicht\cite[S. 39]{Graf}. So ist der Direktverkauf von Marken erfolgreicher denn je, denn sie bieten im Vergleich zu Online-Marktplätzen neben einer besseren Auswahl auf einem bestimmten Gebiet oft deutlich bessere Produktbeschreibung mit größerer Informationsdichte - sprich, bei Produkten einer Brance werden oft auch Anbieter genutzt, die sich auf diese spezialisieren\cite[S. 18f]{evilcom}.

\end{folding}

\begin{folding} % ALLGEMEIN

Das Konsumverhalten hat sich auch unabhängig von der Handelsstruktur geändert: statt gleichbleibenden, rationalen Käufen und Kaufmotiven, die die Auswahl der gekauften Güter stark abhängig von der zur Verfügung stehenden Geldmenge machten\cite[S. 38]{Schramm}; herrscht heute ein deutlich dynamischeres Kaufklima:
\begin{quote}
"So beziehen jetzt zum Beispiel auch solvente Kunden ihre Lebensmittel aus dem Billigdiscounter, während  umgekehrt  einkommensschwächere  Schichten  zu  Luxusgütern  greifen."\cite[S. 43]{Nitt}
\end{quote}
Außerdem gibt es kaum noch "pure" Einzelhändler und Onlinehändler – meist sind Firmen in mehreren Bereichen vertreten, um ihre Präsenz zu steigern. So eröffnet etwa Amazon in den letzen Jahren Läden vor Ort und stationäre Händler betreiben Online-Shops\cite[S. 50]{Graf}.
Außerdem ist das Konsumverhalten in Deutschlands seit Jahren durch eine starke Kaufkraft geprägt, z. T. dank dem 0\%-igen Leitzins der \ac{EZB}\cite[S. 49]{Ebert} – auch wenn die derzeitige Corona-Situation diese leicht abgeschwächt hat\cite{BfWE}. 

\end{folding}



\iffalse 

        EVILCOM --------------------------------------------------------------
        
        auch ältere kaufen online ein: evilcom S 8
        
        durchschnittlicher warenkorbwert steigt, zeichen für mehr akzeptanz des online-shoppings in deutschland. ec S 9
        
        beratung verliert an bedeutung S17
        
        zu folgen?:
        argument retouren und mehr verkehr: retouren trifft nur auf mode zu \s. 25
        online ist nicht umweltschädigender als stationär, weil lieferdienst kunden direkt nacheinander bedienen kann und kunden hin/zurück fahren müssen. modell von schneider et al kommt auf etwa 89% kraftstoff-ersparnis. hier wird aber nicht beachtet, dass kunden meist mehrfach pro fahrt in die stadt einkaufen. - eigene modellrechnung weil besser? \S 25f 
        modifizieren: mehrere geschäfte, größere entfernung weil wir land untersuchen"phänonäm einkaufzentrum wurde nicht beachtet"
        
        pro-arbeiter-umsatz: stationär 210000 amazon 710000: weniger arbeiter werden bei gleichem konsumverhalten benötigt. jedoch ist amazon deutlich effektiver als die meisten anderen anbieter. trotzdem könnte amazon den effizienztrend weiter antreiben \s. 27 werden logistikdienste nicht als eigene mitarbeiter bei amazon gezählt: die zahl ist also geringer ``nur mit vorsicht zu genießen''
        
        weniger arbeitsplätze vor allem in der zukunft wegen effizienzsteigerungen \S. 28
        
        
        NITT----------------------------------------------------------------
        
        verkaufsflächen steigen beim einzelhandel https://de.statista.com/statistik/daten/studie/462136/umfrage/verkaufsflaeche-im-einzelhandel-in-deutschland/: ist jedoch nicht mit wachstum gleichzusetzen. bezieht sich auf gesamte fläche, nicht nur auf die teueren innenstadtflächen. demnach ist eine umlagerung zu flächen am stadtrand denkbar.
        
        --> unsatz pro fläche (flächenproduktivität) sank bis auf die letzten jahre konstant. diese änderung könnte bedueten, dass der einzelhandel wieder fuß fassen konnte (steigende mieten?). ist jedoch mit vorsicht zu genießen , denn das verkaufen von wenig genutzten grundstücken führt auch zu einer erhöhung der flächenproduktivität:https://de.statista.com/statistik/daten/studie/214701/umfrage/flaechenproduktivitaet-im-deutschen-einzelhandel/
        denn gesamtumsätze fallen seit 1990 stetig\cite[S. 6]{Nitt}
        
        in vielen bereichen verlagern sich stationäre händler in (größere) städte, weg von Land und Kleinstädten. jedoch gibt es auch branches mit wenig veränderung, zb nahrungsmittel
    
        folgen auf tühringen bezogen:
        tühringen = sehr schlechte kaufkraft + bevölkerungsanzahl = chancen für unternehmen s 29. schlecht weil sonst kaufen konsumenten am arbeitsplatz s 30
        außerdem nimmt bevölkerung ab s 32f, jedoch nicht gleichmäßig, so nimmt sie zb in münchen/hamburg zu, in tühringen jedoch nicht, tühringen ist im vergleich zu anderen regionen besonders stark betroffen
        
        
        pendler und angestellte kaufen oft nicht in ihrem heimartort sondern am arbeitsplatz ein S 58
    
    
    
    
    
    
        
        allgemein online: großes wachstum ab 2002 S. 26
            jüngere kaufen mehr online ein S 35f
            kleinere haben wenig chancen, da große wie amazon durch hohes kapital niedrige preise finanzieren können, was einer der wichtigsten faktoren beim onlinehandel ist S 37f
            dropshipping: verkauf über händler, versand direkt von hersteller: lagerkosten: n.a.
            datenbeschaffung: bessere produktempfehlungen, infos über kaufverhalten und mögl. dynamische preise
            
    
        
        Stationär----------------------------------------------------------------


            einzelhandel weniger preissensibel; jedoch ist onlinehandel vor allem in der Elektronikbranche konkurrenzfähig, sh. Media Markt\cite[S. 21f]{Graf}

        gute bereiche: essen und beratungsintensives
        gründe, stationär zu kaufen: beratung, feeling
            deutliche verluste ab 2019 S 31
            
            erste onlnie-shops S 30
            
            lebensmittel fast ausschließlich stationär: aber "Picnic" fährt mit robotern die innenstädte ab: S 51


            
\fi








\iffalse
> billig - strategie vorallem von amazon und alibaba

veranschaulichung eigene tabelle börsenwert einzelner unternehmen

Dabei gibt es auch unternehmen, die mehrere berieche nutzen, wie amazon
\fi





%ergebnisse: kataloghandel kann kaum noch bestehen



\iffalse WIRTSCHAFT
allgemein:
        push ist obsolet > pull S 51
        
 Hersteller: sogenannte "Powerseller" kaufen produkte direkt von herstellern, verkaufen sie deutlich unter marktpreisen weiter -> probleme mit preisverfall S. 25
        schrecken aufgrund von biosherigen vertreib über fachhandel meist davor zurück, online-marktplätze einzurichten, vereinzelt(hugoboss.com) S. 30
            "wechseln von b2b zu b2c, sprich bieten waren direkt an, preisvorteil" für billigere preise
        stärker ab 2012, aber powerseller problem S 39f
zuerst ist onlinehandel herausforderung, da "Preis und Leistung der Online-Verkäufer sind praktisch nicht kontrollierbar und undurchsichtig für Marken und Hersteller. Außerdem fürchten die Hersteller einen Kontrollverlust des Markenauftrittes und negative Imageeffekte." S 20



       online----------------------------------------------------------
            hohe retourkosten, neueinsteiger haben es schwierig S 46f
\fi

            
        \subsection{Einfluss des Onlinehandels}% Überschneidung mit 7.2
            \input{exports/Fabian/Konsum/EinflussDesOnlinehandels.tex}
            
        \subsection{Mögliche Probleme und Folgen}
            %zusammenfassung

%gegenseitige beeinflussung: showrooming-effekt: kauf online aufgrund von stationärere beratung -- kauf stationär nach onlinerecherche \cite[S. 21f]{evilcom}

%wie viele läden haben in schleusingen einen onlineshop?


\begin{folding} %stationärer Handel

Aufgrund der Verschiebung von Nachfrage und Bedürfnissen von Konsumenten in den letzten Jahrzehnten wird der stationäre Handel trotz Multichannel-Versuchen keine einfache Zukunft haben. So kann in vielen Fällen stattdessen direkt von Herstellern gekauft werden, die mittlerweile den Vertriebswegwechsel von \ac{B2B} zu \ac{B2C} weitesgehend hinter sich haben. 
Insbesondere in Tühringen steht es im Vergelich zum Rest Deutschlands dank der Kombination aus schlechter Kaufkraft und niedriger Bevölkerungszahl pro Fläche schlecht um den konventionellen Einzelhandel\cite[S. 29]{Nitt}. Dazu kommen demografische Änderungen, die insbesondere in der Mitte Deutschlands Probleme verursachen: so nimmt die Bevölkerung z. B. in Hamburg, trotz Schrumpfen der Bevölkerungszahl, zu - jedoch nicht in Hildburghausen, einer der Landkreise, die am meisten Bewohner verliert\cite[S. 32f]{Nitt}. 
Zum Glück einiger Vertriebe treffen diese schlechten Chancen nicht auf alle Branchen zu - der Lebensmittelvertrieb hat z. B. kaum Online-Konkurrenz[Umfrage]. Um in den restlichen Geschäftssektoren einen maximale großen Umsatz zu erzielen, sollte der konventionelle Einzelhandel aufgrund der alternden Bevölkerung, die meist noch stationär kauft, vorerst Investitionen für die wachsende Gruppe von Senioren und dementsprechend Erreichbarkeit o. ä. nutzen.

\end{folding}

\begin{folding} %Umweltverschmutzung

Zudem kommt immer öfter das Argument auf, dass der Wechsel zum Onlinehandel umweltschädlich sei, da mehr Lieferfahrzeuge unterwegs sind. Jedoch haben bereits die Autoren des "Evil Commerce [...]"-Buches diese These anhand einer Modellrechnung weitesgehend wiederlegt. Sie berechneten eine 90\%-ige Kraftstoffersparniss bei komplettem Umstieg zum Distanzhandel in Großstädten unter optimalen Bedingungen\cite[S. 25f]{evilcom}. Um genauerer Aussagen bezüglich des ländlichen Raumes zu treffen, werde ich die genannte Rechnung bzgl. Entfernung - da Einkaufszentren nicht in Betracht gezogen wurden - modifizieren und insofern erweitern, dass zusätzlich ein Einkauf in mehreren Geschäften nacheinander mit in Betracht gezogen wird.

\begin{itemize}

\item Im meiner Modellrechnung kaufen 100 Bewohner eines Dorfes in einer 4km entfernten Stadt ein. Sie kaufen im Durchschnitt in 3 von 10 Einkaufsmöglichkeiten ein, die je 500m voneinander entfernt sind. Dabei gehe ich davon aus, dass alle Kunden über eine 500m lange Straße innerhalb des Dorfes zu erreichen sind.

\item Wenn alle Bewohner stationär kaufen, legen sie im Durchschnitt eine Strecke von 

\begin{align}(250m + 4000m + 3 \cdot 500m + 4000m + 250m) \cdot 100 = 1000000m\end{align}

 zurück. Dabei nehme ich an, dass alle Bewohner denselben Ortsausgang benutzen und somit einen durchschnittlichen Weg von 250m zu diesem haben.

\item Wenn jeder Verkäufer jedoch die Güter an seine im Durchschnitt 30 Kunden versendet, müssten alle Lieferwagen zusammen eine Strecke von gerade einmal 

\begin{align}(4000m + 500m + 4000m) \cdot 10 = 85000m\end{align}

zurücklegen.
\item In dieser Darstellung hat der Distanzhandel eine ähnlich hohe Kraftstoffersparniss - 91.5\%. 

\end{itemize}

Zwar kann mithilfe dieses Modelles die These der Umweltverschmutzung auch auf dem Land wiederlegt werden, jedoch ist sie keine genaue Darstsellung der Realität, da viele Faktoren, wie z. B. die Retouranzahl, die beispielsweise in der Modebranche überproportional hoch ist, nicht beachtet wurden(ebd.).

\end{folding}

        \newpage
            
        
    \section{Auswirkungen des Oninehandels auf die Infrastruktur}
        \subsection{Vergangene Änderungen und Entwicklungen}
            \input{exports/Fabian/Infrastruktur-Vergangenheit.tex}
            
        \subsection{Mögliche Entwicklungen in der Zukunft}
            \input{exports/Fabian/Infrastruktur-Zukunft.tex}
            
        \subsection{Probleme mit dem Einfluss des Onlinehandles}
            \input{exports/Fabian/Infrastruktur-Probleme.tex}
            
        
    \section{Globaler Vergleich des Einflusses und der Entwicklung des Onlinehandels}
            \iffalse

    https://www.worldretailcongress.com/__media/Global_ecommerce_Market_Ranking_2019_001.pdf
    
     \cite{esworld} 

    name: globaler vergleich bezüglich des onlinehandels

\fi

Deutschland ist weltweit eines der Länder mit dem größten technischen Fortschritt und ist deshalb auch im Bereich Onlinehandel und -versand mit 63,9 mio. Onlineeinkäufern in 2019 vergleichsweise stark vertreten \cite[S. 8]{esworld}. Doch in welchen Punkten unterscheidet sich die Struktur dessen in einem internationalen Vergleich?

In einer globalen Rangliste von \emph{eshopworld} belegte Deutschland 2019 im allgemeinen Vergleich Platz 5 von 30, nach den USA, China und weiteren \cite[S. 3]{esworld}. Jedoch war Deutschland in vielen Unterpunkten kaum vertreten - bis auf die Kategorie Logistik. Hier belegte es in weiteren 3 Unterkategorien 2 mal den 1. Platz \cite[S. 10ff]{esworld}. Diese Unterkategorien waren einerseits Zölle sowie Logistik allgemein und sind durch die Existenz der Europäischen Union erklärbar. Denn ähnlich wie in den USA \cite[S. 4]{esworld} sind Verkäufe, die Landesgrenzen übergreifen, hier nur mit einem geringen Aufwand möglich - z. B. dank niedriger Zölle innerhalb der EU.%quelle?

Zusätzlich ist Deutschland nach Austritt der UK der größte Onlinemarktpatz Europas und durch Grenzen an 9 Nachbarländern sowie der relativ fortschrittlichen Infrastruktur besonders attraktiv für Onlineanbieter. Vor allem neue Verkäufer können daraus einen großen Vorteil ziehen, da Markteintrittsbarrieren\footnote{Hindernisse, die Unternehmen daran hindern, sich am Markt zu etablieren} so deutlich niedriger sind \cite[S. 8]{esworld}.

Deutschland hat sich in den letzten Jahrzehnten dank seiner Lage und progressiven\footnote{fortschrittlich} Infrastruktur trotz einigen Hindernissen zu einem der größten Onlinemarktplätzen weltweit entwickelt. Besagte Hindernisse sind z. B. die vergleichsweise kleine Landfläche und das 14-tägige Rückgaberecht von Artikeln ohne Angabe eines Grundes, die online bestellt werden(§355 BGB). Letzteres ist ein besonders schwerwiegendes Problem vor allem für kleinere Unternehmen, da jede Rücknahme einen verhältnismäßig hohen Verlust darstellt \cite{retourwahnsinn}.

        \newpage
            
    
    \section{Onlineversandhändeler am Beispiel Amazon} 
        Die Zusammensetzung der Güternachfrage hat sich in den letzten Jahrzehnten stark verändert. Insbesondere unter dem Aspekt des Onlinehandels ist es nun für Firmen wichtig, ihre Verkaufskonzepte evt. zu erweitern oder zu aktualisieren. Im folgenden Teil werde ich die Entwicklung sowie das aktuelle Kaufverhalten von Konsumenten mithilfe einer [Umfrage] analysieren und auf Basis der Ergebnisse möglich Folgen formulieren.

        \subsection{Entstehung und Entwicklung}
            Als Amazon, anfangs noch \emph{cadabra.com}, am 5. Juli 1994 von Jeff Bezos und seiner Frau McKenzie gegründet wurde, hatte wahrscheinlich niemand die Vision eines marktführendem Online-Unternehmens im Kopf - im Gegensatz, Amazon war ursprünglich ein Online-Buchhandel für bestimmte, seltene Bücher\cite[S. 17]{Graf}. Trotz der kleinen Zielgruppe wuchs das Unternehmen in den folgenden Jahren bedeutend: schon zwei Jahre später wurden Aktien angeboten, außerdem wurde anfangs noch fast der komplette Gewinn reinvestiert\cite{Rosoff}, was das Aufkaufen ganzer Unternehmen schon 4 Jahre nach der Gründung ermöglichte, bespielsweise von \emph{pets.com} und \emph{overstock.com}\cite{ChannelAdvisor}. Mit der Zeit expandierte die Firma in viele weitere Gebiete: Cloud Computing mit \ac{AWS} 2002 sowie Musik mit einem Online-Musik-Store und Lebensmittel mit AmazonFresh im Jahr 2007\cite{Sherman, ChannelAdvisor}. Auch bezüglich des Onlinehandels breitete Amazon ab 2000 nach und nach die Produktauswahl aus, wodurch sich der darmalige Buchhandel zu dem heutigen Onlineversandhandel für fast alle Produktereiche entwickelte. Ein wichtiger Schritt zu diesem Ziel war das Ermöglichen von Drittanbieter-Verkäufen ab dem 30. September 1999, was die Bekanntheit und Anzahl der Verkäufe erheblich steigerte\cite{Sherman}. Zudem ermöglich es der Firma, ein breiteres Produktsegment anzubieten sowie Provisionen zu erhalten, während Probleme wie Kapitalbildung und Lagerplätze an dritte Händler ausgelagert werden\cite[S. 50]{evilcom}. Außerdem wurden weitere Technologien wie Amazon Prime und AmazonBasics entwickelt, die den Onlinehandel und -versand unterstützen\cite{ChannelAdvisor}, aber auch alleinstehende Projekte, wie Kindles, das Fire Phone oder Smart-Home-Geräte\cite{Sherman}.
Die derzeitige Strategie bezüglich des Oninehandels beschrieb Bezos, als "Virtuos Cycle" betitelt, schon 2001  mit folgender Zeichnung\cite{zentail}:

%IST NICHT IMMER DA WO ES SOLL WENN ES ZB NICHT AUF DIE SEITE PASST
\begin{figure}[h]
    \begin{center}
        \includegraphics[width=8cm]{media/Fabian-vicious-cycle.png}
        \caption{Amazon's Vicious Cycle}
        \label{vicious-cycle}
        \bildquelle Jeff Bezos, September 2001 %Learn from the Bezos Virtuous Cycle: Leverage and Invest in Infrastructure, www.zentail.com, abgerufen August 2020% https://tinyurl.com/yyu2zz29 DATUM???
    \end{center}
\end{figure} 

Dabei schafft breit gefächerte Produktsegment(Selection) eine positive Kundenerfahrung(Customer Experience), die weitere Verkäufe und Verbreitung durch z. B. Empfehlungen(Traffic) hervorruft. Durch diese hohe Kundenanzahl ist die Plattform wiederrum attraktiver für Drittanbieter und Herstellern(Sellers), die weitere Produkte anbieten und so das Produktsegment erweitern. Dieser Teil ist an sich nicht wirklich außergewöhnlich, da viele andere Onlineanbieter eine ähnliche Strategie verfolgen. Jedoch hebt sich Amazon damit ab, ungewöhnlich hohe Summen zu investieren, um Kosten(Lower cost structure) und somit auch Produktpreise(Lower prices) zu senken\cite[S. 26f]{Graf}. Amazon schaffte so auch ein neues Konsumverhalten, das "Amazon Commerce" - Graf und Schneider beschreiben es in ihrem Buch als ein

\begin{quote}
    "[...] komplett neues Kaufverhalten, das sich nicht mehr an Anbietern oder konkreten Produkten orientiert, sondern allein am Zweck [...], den das gewünschte Produkt erfüllen soll."\cite[S. 42]{Graf}
\end{quote}
Die genannten Punkte ermöglichten es Amazon, sich als weltweit bekannten und benutzten Onlineversandhandel zu etablieren - jedoch haben sie auch einige Probleme hervorgerufen. Beispielsweise führte die konstante Niedrigpreispolitik\cite[Abb. 5]{Desjardins} zum Einsparen von Ausgaben in fast allen Gebieten - auch im Bezug auf Angestellte\cite[S. 6]{Apicella}. So werden insbesondere in der Weihnachtszeit Leiharbeiter eingestellt. In der ARD-Reportage "Ausgeliefert! Leiharbeiter bei Amazon" wird 2013 gezeigt, wie deren Arbeitsalltag aussah: Zu siebt wird in einer Ferienwohnung übernachtet, oft bekommen die Angestellten nur wenige Stunden Schlaf. Jeden Tag aufs neue ist es unsicher, ob man gebraucht wird - wenn nicht, gibt es keinen Lohn. Mitarbeiter der Dienstleistungsgewerkschaft Ver.di und Amazons erklären, dass 2013 in Koblenz circa 3100 von 3300 Arbeitern befristet angestellt waren\cite{Ausgeliefert}.
Außerdem existiert ein hoher Grad an Überwachung und Kontrollen, wie Apicella in ihrer Studie andhand der Stadt Leipzig beschreibt:

\begin{quote}
    "Die Verkaufsarbeit durchläuft dabei einen Prozess der [...] vollständige[n] Überwachung und Disziplinierung der Beschäftigten[...].\cite[S. 29]{Apicella}"
\end{quote}
Dementsprechend sind Arbeitseinstellungen bei Amazon keine Seltenheit: Beispielsweise streikten Angestellte in Deutschland zwei Monate nach der besagten Reportage unter dem Motto "Wir sind keine Roboter" gegen niedrige Löhne, befristete und allgemein schlechte Arbeitsverhältnisse sowie die starke Digitalisierung der Arbeit\cite[S. 6]{Apicella}. Amazon reagierte in den folgenden Jahren mit mehreren Lohnerhöhungen, jedoch exestieren noch vereinzelt Streiks, da die Arbeitsbedingungen anscheinend immer noch problematisch sind\cite{JGraf}. So schrieb Amazon z. B. 175000 neue Stellen in Folge der Corona-Krise und einem 32\%-igem Verkaufszuwachs aus - nicht nur, weil mehr Arbeiter als vorher gebraucht werden, sondern auch weil einige Angestellte aufgrund von "unsicheren Bedingungen" sich weigerten, zu ihrem Arbeitsplatz zu erscheinen\cite{Theweek}.

Innerhalb der letzen 26 Jahre hat Amazon sich von einem Online-Buchhandel zu einem weltweiten Onlinehändler fast alle Produktklassen entwickelt. Außerdem bietet die Firma heute auch andere Dienste an, wie z. B. Cloud Computing mit \ac{AWS}. Jedoch steht das Unternehmen bezüglich der Arbeitsbedingungen seit fast einem Jahrzehnt in der Kritik.

            
        \subsection{Einfluss auf das Konsumverhalten } %neu: das Konsumverhalten und der Einfluss des Onlinehandels
            
Außerdem führt das seit über einem Jahrzehnt steigende und von Großfirmen wie Amazon angetriebene Wachstum des Onlinehandels zu einem Rückgang der Nachfrage im stationären Handel\cite{Shankar} - mehr dazu in Punkt [8].

\iffalse
 alles einfacher und unkompliziert

 Vorreiter in sachen niedrige Preise > ist sehr wichtig, weil
   viel einfacher vergelichbar, qualität des Produkts nicht einfach einsehbar: sie muss nicht außergewöhnlich, nur akzeptabel sein - jedoch auch nicht schlecht, da 14-tage-rückgabe ohne angabe eines grundes

 einfluss extrem in coronazeiten

 S 49 https://edoc.sub.uni-hamburg.de/hcu/volltexte/2017/370/pdf/Ebert_Kirsten.pdf
 danach: modell für veränderung


Amazon hat als Vorreiter bezüglich niedriger Preise das Verhalten von Konsumenten stark beeinflusst:


Amazon-bezogen: 

    Niedrige Preise + Verfügbarkeit, auswahl -> amazon reicht als einzige einkaufsmöglichkeit
    schneller versnad mit prime - amazon macht verlust
    
\fi

            
        \subsection{Onlinehändler als Konkurrenten zu lokalen Händlern}
            

\iffalse
    AMAZON-BEZOGEN

billig-entwicklung, konkurrenz muss kosten einsparen -> ausbeutung arbeiter, erdrängung kleinerer Onlinhändler und Geschäfte im ländlichen Raum.? kann man die Allgemeinheit auf Schleusingen übertragen

quelle sha19: 6000 stationre schließungen usa





ergebnisse: retouren machen großteil der ausgaben aus
\fi

%ergebnisse: auf branche bezogen; lokale händer bestellen artikel online + beratung für höhere auswahl: aber kein direktes betrachten

        \newpage
            
        
    \section{Wirtschaft} %mit blick auf anbieter: sonst überschneidung mit konsumverhalten
        \subsection{Entwicklung der Wirtschaft}
        \subsection{Auswirkungen auf Konzerne und Umwelt}
        \subsection{Auswirkungen auf Paketdienste}
        \subsection{Folgeänderungen von Import/ Export}
        \newpage
        
    \section{Handeln im Internet}
        \subsection{Marketing}
             % https://books.google.de/books?hl=de&lr=&id=KpvzBQAAQBAJ&oi=fnd&pg=PR5&dq=konsumverhalten+entwicklung&ots=7XtmPqXXmZ&sig=xCkMsXi7Up7jXC479QNXaTYPz3o&redir_esc=y#v=onepage&q=konsumverhalten%20entwicklung&f=false

        \subsection{Bezahlvorgänge}
            \input{exports/Fabian/Handeln-Bezahlvorgänge.tex}
        \newpage
    \bibliography{Literaturverzeichnis}
        \newpage
    \listoffigures
        \newpage
    \addsec{Abkürzungsverzeichnis}
\label{sec:abkuerzungsverzeichnis}

\begin{acronym}[AWS] % in [] die längste akürzung
    \acro{AWS}{Amazon Web Services}
    \acro{B2B}{Bussiness-to-Bussiness}
    \acro{B2C}{Bussiness-to-Consumer}
    \acro{EZB}{Europäische Zentralbank}
    \acro{Obm}{Onlinebezahlmethode}
    \acro{BIM}{Building Information Modelling}
\end{acronym}
\newpage

        \newpage
    \listoftables

\end{document}


%ergenisse: nischenwaren mit onlineversand
    %viele einzelhandelsflächen ini der stadt durch insolvenzen
    %mglich: hersteller-fabrik einige km von stadtzentrum, laden in der stad, mit onlineshop
