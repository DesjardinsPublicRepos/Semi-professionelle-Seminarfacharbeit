%?: mode beetz, ..., in schleusingen schließen immer mehr läden und es werden kaum neue eröffent(abgesehen von restoraunnts). gleichzeitig wird der onlineeinkauf immer attraktiver


%Mode Beetz, ... - in Schleusingen schließen immer mehr Läden, währen kaum neue eröffnet werden. Gleichzeitig wird der Onlineeinkauf immer attraktiver. Da kommt schnell die Befürchtung auf, dass die Innenstadt verschwinden wird. 


%tühringer Landkreis dafür bekannt, dass er technisch hinterherhängt -> wie sieht das in anderen bundesländern aus und was können wir daraus lernen?
%wie in zukunft
Mode Beetz, ein Geschenkeladen, ein Bäcker sowie ein Fleischer am Markt - das Teehaus in der alten Burgstraße 9 - ein Zeitschriftenhandel und ein Jeansladen beim Megacenter - und viele weitere stationäre Läden schlossen in Schleusingen im Verlauf der letzten Jahre. Parallel dazu spielt der Onlinehandel eine immer größer werdende Rolle im Leben von Menschen fast aller Altersschichten. Aufgrund dieser Tatsachen haben wir uns die Frage gestellt, welche Rolle der Verkauf von Waren über das Internet in unserem Landkreis spielt und ob die zunehmenden Schließungen einiger Geschäfte auf einen durch Distanzhandel begründeten Rückgang der Nachfrage für lokale Einzelhändler zurückzuführen ist.\\

\noindent Im Rahmen unserer Arbeit werden wir das Zutreffen folgender Thesen einschätzen:
\begin{itemize}
    \item Der Onlinehandel führt zu einem allmählichen Aussterben von stationären Händlern im ländlichen Bereich.
    
    \item Es gibt Waren, die nicht online gekauft werden.

    \item Der Distanzhandel schafft weniger Arbeitsplätze als indirekt verringert werden.
    
    \item Das zunehmende Wachstum des Onlinehandels ist eine Belastung für die Umwelt.\\
\end{itemize}

\noindent Außerdem werden wir auf Basis unserer Ergebnisse ein Konzept entwickeln, auf welche Besonderheiten lokale Händler achten sollten, um eine erhöhte Nachfrage zu erfahren - und für welche stationäre Unternehmen nahezu keine gewinnbringende Zukunft mehr möglich ist.
