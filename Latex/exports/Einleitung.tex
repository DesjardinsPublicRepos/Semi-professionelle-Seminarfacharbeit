%?: mode beetz, ..., in schleusingen schließen immer mehr läden und es werden kaum neue eröffent(abgesehen von restoraunnts). gleichzeitig wird der onlineeinkauf immer attraktiver



Onlinehandel spielt eine immer größer werdende Rolle im Leben von Menschen fast aller Altersschichten. Aufgrund des allmählichen Verschwindens von Geschäften im Gebiet um Schleusingen haben wir uns die Frage gestellt, welche Rolle der Verkauf von Waren über das Internet in unserem Landkreis spielt, und ob die Schließungen einiger Geschäfte auf den Rückgang der Nachfrage für lokale Einzelhändler zurückzuführen ist.

Im Rahmen unserer Arbeit werden wir das Zutreffen folgender Thesen einschätzen:
\begin{quote}
    \RM{1} Der Onlinehandel führt zu einem allmählichen Aussterben von stationären Händlern im ländlichen Bereich.
\end{quote}

\begin{quote}
    \RM{2} Es gibt Waren, die nicht/kaum online gekauft werden.
\end{quote}

\begin{quote}
    \RM{3} Der Onlinehandel schafft weniger Arbeitsplätze als indirekt verringert werden.
\end{quote}

Außerdem werden wir auf Basis unserer Ergebnisse ein Konzept entwickeln, wie die Nachfrage und der Verkauf von Waren in unserem Landkreis erhöht werden kann.
