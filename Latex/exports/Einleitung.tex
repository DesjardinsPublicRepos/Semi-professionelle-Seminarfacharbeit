%?: mode beetz, ..., in schleusingen schließen immer mehr läden und es werden kaum neue eröffent(abgesehen von restoraunnts). gleichzeitig wird der onlineeinkauf immer attraktiver


%Mode Beetz, ... - in Schleusingen schließen immer mehr Läden, währen kaum neue eröffnet werden. Gleichzeitig wird der Onlineeinkauf immer attraktiver. Da kommt schnell die Befürchtung auf, dass die Innenstadt verschwinden wird. 


%tühringer Landkreis dafür bekannt, dass er technisch hinterherhängt -> wie sieht das in anderen bundesländern aus und was können wir daraus lernen?
Onlinehandel spielt eine immer größer werdende Rolle im Leben von Menschen fast aller Altersschichten. Aufgrund des allmählichen Verschwindens von Geschäften im Gebiet um Schleusingen haben wir uns die Frage gestellt, welche Rolle der Verkauf von Waren über das Internet in unserem Landkreis spielt, und ob die Schließungen einiger Geschäfte auf den Rückgang der Nachfrage für lokale Einzelhändler zurückzuführen ist.

Im Rahmen unserer Arbeit werden wir das Zutreffen folgender Thesen einschätzen:
\begin{itemize}
    \item Der Onlinehandel führt zu einem allmählichen Aussterben von stationären Händlern im ländlichen Bereich.
    
    \item Es gibt Waren, die nicht online gekauft werden.

    \item Der Onlinehandel schafft weniger Arbeitsplätze als indirekt verringert werden.
    
    \item Das zunehmende Wachstum des Onlinehandels ist eine Belastung für die Umwelt.
\end{itemize}

Außerdem werden wir auf Basis unserer Ergebnisse ein Konzept entwickeln, wie die Nachfrage und der Verkauf von Waren in unserem Landkreis erhöht werden kann.
