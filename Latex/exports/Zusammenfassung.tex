 
\subsubsection{Einschätzung der Thesen}

Die Existenz des Distanzhandels verringert bereits seit seiner Entstehung die Nachfrage des stationären Handels in fast allen Produktgruppen, und damit zu stationären Umsatzeinbußen. Infolgedessen haben stationäre Händler - insbesondere im ländlichen Bereich, da hier aufgrund einer niedrigeren Bevölkerung pro Vertrieb eine allgemein geringere Nachfrage vorherrscht - deutlich schlechtere Chancen, ihr Geschäft Umsätze erzielen zu lassen, insbesondere falls Gütergruppen, die nicht in die Produktgruppe der \ac{FMCG} fallen, verkauft werden. Dementsprechend können derzeit deutlich mehr Insolvenzen stationärer Vertriebe beobachtet werden und die These "Der Onlinehandel führt zu einem allmählichen Aussterben von stationären Händlern im ländlichen Bereich." verifiziert werden.

Jedoch muss dabei auch bzgl. der Gütergruppen differenziert werden - zwar können alle Güter mit mehr oder weniger Nachteilen online gekauft werden, jedoch ist der Aufwand und weitere negative Einflüsse des Onlinehandels bei einigen Produkten so relevant, dass diese fast ausschließlich stationär gekauft werden, wie beispielsweise Nahrungsmittel. Dementsprechend kann auch die These, dass es Waren gibt, die nicht online gekauft werden, bis auf einige Ausnahmefälle bestätigt werden.

Bei Betrachtung der Arbeitsverhältnisse ist eine eindeutige Einordnung der der These, dass durch den Onlinehandel weniger Arbeitsplätze geschaffen werden als indirekt verringert werden, kaum möglich. Einerseits verlieren stationäre Händler an Relevanz und können folglich weniger Arbeitnehmer beschäftigen, jedoch werden, un die Funktionalität des Distanzhandels zu gewährleisten, auch mehr Arbeiter gesucht - insbesondere für Lager- und Transportverwaltung. Außerdem ist das Aussterben von Innenstädten nicht ausschließlich durch den Onlinehandel begründbar, sondern zu einem großen Teil auch durch eine 
\begin{quote}
"grundsätzliche Verschiebung in den Handelsmodellen"\cite{evilcom}.
\end{quote}
Demnach kann die genannte These werder falsifiziert noch verifiziert werden.

Die vierte These, die im Rahmen dieser Arbeit untersucht wird, beschreibt dass durch den Onlinehandel eine zunehmende Umweltbelastung entsteht. Dies ist in erster Linie anhand einem erhöhtem Verkehrsaufkommen durch dass paralele Nutzen von Online- und Offlinehandel problemlos verifizierbar - jedoch könnte in Zukunft, falls nahezu ausschließlich der Vertriebsweg des Distanzhandels genutzt wird, sogar eine Kraftstoffersparnis, die in Kapitel [4.1.2] beschrieben wird, erreicht werden.
