\iffalse

deu: 14-tage rückgabe durch §355 BGB

https://www.worldretailcongress.com/__media/Global_ecommerce_Market_Ranking_2019_001.pdf
\cite{esworld} 

--> 
    63.9mio online shoppers, D ist ein land in dem es den onlinehandel schon lange gibt
    
<--


einleitung: schlüsse aus ländern, die onlinehandel stärker nutzen, ziehen; jedoch ist D. schon vorreiter also wenig info
\fi

%name: globaler vergleich bezüglich des onlinehandels

Deutschland ist weltweit eines der Länder mit dem größten technischen Fortschritt und ist deshalb auch im Punkt Onlinehandel vergleichsweise stark vertreten. Doch in welchen Punkten unterscheidet sich die Struktur dessen in einem internationalen Vergleich?

In einer globalen Rangliste von \emph{eshopworld} belegte Deutschland 2019 im allgemeinen Vergleich Platz 5 von 30, nach den USA, China und weiteren\cite[S. 3]{esworld}. Jedoch war Deutschland in anderen Unterpunkten kaum vertreten - bis auf die Kategorie Logistik. Hier belegte es in weiteren 3 Unterkategorien 2 mal den 1. Platz\cite[S. 10ff]{esworld}. Diese Unterkategorien sind Logistik allgemein sowie Zölle und durch die Existenz der Europäischen Union erklärbar. Denn ählich wie in den USA\cite[S. 4]{esworld} sind Landesgrenzen übergreifende Verkäufe hier mit nur geringem Aufwand möglich - z. B. dank niedriger Zölle innerhalb der EU.%quelle?

Zusätzlich ist Deutschland nach Austritt der UK der größte Onlinemarktpatz Europas und durch Grenzen an 9 Nachbarländern sowie der relativ fortschrittlichen Infrastruktur besonders attraktiv für Onlineanbieter. Vor allem neue Verkäufer können daraus einen großen Vorteil ziehen, da Markteintrittbarrieren so deutlich niedriger sind\cite[S. 8]{esworld}.

Deutschland hat sich in den letzten Jahrezehnten dank seiner Lage und progressiven\footnote{fortschrittlich} Infrastruktur trotz einigen Hindernissen zu einem der größten Onlinemarktplätzen weltweit entwickelt. Besagte Hindernisse sind z. B. die vergleichsweise kleine Landfläche und das 14-tägige Rückgaberecht von Artikeln ohne Angabe eines Grundes, die online bestellt werden(§355 BGB). Letzteres ist ein besonders schwewiegendes Problem vor allem für kleinere Unternehmen, da jede Rücknahme einen verhältnismäßig hohen Verlust darstellt\cite{retourwahnsinn}.

%trotz einschränkuungen wie bgb stark vertreten
