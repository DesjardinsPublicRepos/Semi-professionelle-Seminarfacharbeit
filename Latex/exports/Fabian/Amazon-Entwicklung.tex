Amazon ist ein Onlineversandhändler, der eine breite Bekanntheit genießt und vor allem im europäischen Raum im Bereich des Onlinehandels einen großen Anteil ausmacht. Während Aliexpress in Asien und östlichen Teilen Europas und Ebay im Norden der EU stark vertreten sind, ist Amazon in Zentral- und Nord/ Westeuropa die Plattform mit den Größten Wert von Verkäufen \cite[S. 22]{EuroCommerce}. Außerdem macht Amazon bereits seit Ende 2017 mehr als die Hälfte der Verkäufe durch Drittanbieter aus\cite[S. 25]{Haendlerbund}. Dementsprechend werde ich in den folgenden Unterpunkten die Firma in Bezug auf ihre Entwicklung, die Auswirkungen dieser und der Konkurrenzfähigkeit zu lokalen Händlern analysieren und auf Basis der Erkentnisse Schlussfolgerungen im Bezug auf Schleusingen und des hildburghäuser Landkreis schließen.

Als Amazon am 5. Juli 1994 von Jeff Bezos und seiner Frau McKenzie gegründet wurde[], hatte wahrscheinlich niemand die Vision von einem marktführendem Online-Unternehmens im Kopf. Ganz im Gegensatz, Amazon war ursprünglich ein Online-Buchhandel[] für bestimmte, seltene Bücher.Trotz der kleinen Zielgruppe wuchs das Unternehmen in den folgenden Jahren bedeutend. Beispielsweise wurde am 15. Mai 1997 wurden zuerst Aktien zum Verkauf freigegeben und 5 Monate später eine zweite Verteilstelle eröffnet.

Im Juli 1998 expandierte der Buchhandel schließlich in anderes Gebiet, den Verkauf von Musik, mit einer sehr Großen Anfangsmenge von 125000 Titeln. 
Am 28. September desselben Jahres meldete Amazon ein Patent für ihre "1-Click Checkout"-Funktion an, was andere Onlinehändler bis zum Verfall des Patentes 2017 dazu zwung, die Funktion lizensieren zu lassen.
Noch im selben Monat wurde es außerdem Drittanbietern ermmöglicht, ihre Waren zum Verkauf einzustellen, was die Bekanntheit und Menge der Verkäufe in hohen Maße steigerte.
Ab der Jahrtausendwende weitete sich die Auswahl von Waren weiter aus, z. B. auf Elektronikgeräte und Anziehsachen. Außerdem wurede AWS (Amazon Web Services) gestartet.
In den folgenden Jahren kaufte Amazon auch andere Firmen wie z. B. joyo, Audible, Zappos, Kiva Systems und Twitch; einerseits um neue Verkaufsbereiche zu erschließen, aber auch um mögliche Konkurrenz unschädlich zu machen.
Noch wichtiger sind aber neue Technologien, die Amazon ab 2004 entwickelte, beispielsweise Kindles, Amazon Prime, das Fire Phone, und vor allem Smart-Home Geräte wie den Echo oder Alexa.

/*
Amazon's Haupteinnahmequelle ist zweifelsfrei der Onlinehandel und -versand - und genau dieser hat insbesondere Heute bedeutende Probleme. 
  >billg-wettbewerb: amazon macht tlw. verlust, um andere zu verdrängen, zB Prime Versand an einem Tag
  >verbilligung: Lohn und Arbeitsbedingungen schlecht "verbilligung nicht nur auf kosten des Gewinns, sondern auch der Arbeiter"



>CORONA, besordere Artikel wie Toilettenpapier oder desinfektionsmittel

>WEGSCHMEI?EN ABGELAUFENER LEBENSMITTEL > onlinehandel wenoger sclimm weil spielraum
-> TAUSCH REINFFOLGE   AMAZON - AUSWIRKUNGEN ONLINEHANDEL AUF INFRASTRUKTUR
DIE IN 5.1 BESCHRIEBENE AUSBEUTUNG UND VERBILLIGUNG WURDE GRÖ?TENTEILS DURCH AMZON HERVORGERUFEN


[CNN]
[Wikipedia]

>Ausnutzung
>Kontakt mit Amazon

>Schlussfolgerungen: auf Tühringen übertragbar, weil 

9.3 Schlussfolgerungen*/
