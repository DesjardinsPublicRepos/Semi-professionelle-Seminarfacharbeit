\subsubsection{Konsumentenperspektive}
Aus Konsumentenperspektive hat der stationäre Handel kaum Vorteile gegenüber dem Distanzhandel, wodurch die stationäre Nachfrage nach Gütern, die paralel online erhältlich sind über die nächsten Jahre - insbesondere unter Einfluss der Corona-Pandemie\footnote{eine Krankheit, die sich weltweit verbreitet} - weiterhin in großen Ausmaß fallen wird. Insbesondere infolge des Wegfallens des Weihnachtsgeschäftes in Deutschland sind weitere Insolvenzen kaum vermeidbar.

Bestehende Händler und Hersteller können jedoch mit Veränderungen der Verkaufsstruktur dagegen ankämpfen und die Änderungen vollbringen, die schon vor Jahren hätten umgesetzt werden müssen[?]: Einerseits brauchen nun viele stationäre Händler zwingend weitere, potente\footnote{stark, mächtig} Vertriebswege, wobei sich der Distanzhandel anbietet - insbesondere wenn das Unternehmen Produkte außerhalb des Güterbereiches der \ac{FMCG} vertreibt. Hersteller sollten dagegen - zumindest die, die es bereits noch nicht getan haben - schnellstmöglich von dem \ac{B2B}- zu dem \ac{B2C}-Vertriebsmodell wechseln, da der Direktverkauf große Ersparnisse mit sich bringt und heutzutage durch die zunehmende Digitalisierung nahezu problemlos möglich ist.

Für Firmen, die neu am Markt sind bietet sich der Distanzhandel auch als primärer oder sekundärer Vertriebsweg an - so kann er genutzt werden, um mögliche stationäre Verluste in der Zukunft auszugleichen oder um die Gesamtgewinnd zu steigern, da eine größere Kundenmenge angesprochen wird. Ein pur stationärer Vertrieb ist zurzeit insbesondere unter Corona-Einfluss meist keine Gewinnversprechende Vertriebsform, vor allem, wenn die besagte Firma noch neu am Markt ist.

Auch je nachdem, was Unternehmen verkaufen unterscheiden sich Zukunftsschancen stark: Ein weiterer Drogerieladen in einer Kleinstadt wird wenig Anklang finden; ein Geschäft mit Nischenorientierung dagegen eher - hier kann die geringe Nachfrage mit einem paralel exestierenden Distanzhandel erhöht werden. %beispiel held der steine als vorzeigebeispiel, natürlich nicht jedem möglich eine nische zu finden, schlussendlich gilt:
Abschließend gilt: Solange es wenig bis keine ähnliche Läden gibt und eine ausreichen hohe Nachfrage vorherrscht, können neue Unternehmen bestehen.

Im Gesamtbild ist zu beobachten, dass der stationäre Einzelhandel einen Gewinnrückgang erlebt - und demzufolge einige Händler insolvent wurden. Jedoch ist dies nicht mit dem Aussterben von Innestädten gleichzusetzen, der oft durch den Onlinehandel begründet wird. Der Distanzhandel senkt zwar die Flächenproduktivität, jedoch weichen nun oft stationäre Händler auf neu erschlossene Gebiete ausßerhalb von Städten aus, da Mieten billiger sind \cite[S. 30]{evilcom}. Die wenigen, die schließen müssen, stellen nur einen kleinen Teil des Einzelhandels dar - der wiederrum nur ein Bruchteil aller Unternehmen ausmacht.

% anpassung an neue bedürfnisse

%ergenisse: nischenwaren mit onlineversand, oder online-verbund mit anderen\ nitt 63
    %viele einzelhandelsflächen ini der stadt durch insolvenzen
    %mglich: hersteller-fabrik einige km von stadtzentrum, laden in der stad, mit onlineshop
    %stärkere, individiuelle bedürfnisse
    %standort immer wichtiger, am besten shoppingcenter \nitt, 7
    %handwerker, baumarkt, apotheken
    %lebensmittel-liefer-roboter zwar in zukunft denkbar, jedoch noch nicht umsetzbar weil probleme wie vandalismus(unbegründete zerstörung von dingen)
    %stationärer handel hat in vielen gebieten geringe zukunftschancen: er wird zwar überleben, trotzdem ist der einstieg neuer händler um so schwerer - jedoch wird er für eine interesannte innenstadt nicht benötigt
    %attraktionen statt verkauf: zB lasertech
    %oder Gastronomie- und Dienstleistungsangebote (zum Beispiel Frisör, Reinigung, Krankengymnastik, Arztpraxen)
    % oder Bildungs- oder Kulturangeboten wie fahrschule
