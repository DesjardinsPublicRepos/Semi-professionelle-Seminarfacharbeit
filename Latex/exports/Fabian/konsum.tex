\subsubsection{Konsumentenperspektive}
asldfjjsdalfjlsdajfldsf
 

%ergenisse: nischenwaren mit onlineversand, oder online-verbund mit anderen\ nitt 63
    %viele einzelhandelsflächen ini der stadt durch insolvenzen
    %mglich: hersteller-fabrik einige km von stadtzentrum, laden in der stad, mit onlineshop
    %stärkere, individiuelle bedürfnisse
    %standort immer wichtiger, am besten shoppingcenter \nitt, 7
    %handwerker, baumarkt, apotheken
    %lebensmittel-liefer-roboter zwar in zukunft denkbar, jedoch noch nicht umsetzbar weil probleme wie vandalismus(unbegründete zerstörung von dingen)
    %stationärer handel hat in vielen gebieten geringe zukunftschancen: er wird zwar überleben, trotzdem ist der einstieg neuer händler um so schwerer - jedoch wird er für eine interesannte innenstadt nicht benötigt
    %attraktionen statt verkauf: zB lasertech
    %oder Gastronomie- und Dienstleistungsangebote (zum Beispiel Frisör, Reinigung, Krankengymnastik, Arztpraxen)
    % oder Bildungs- oder Kulturangeboten wie fahrschule
