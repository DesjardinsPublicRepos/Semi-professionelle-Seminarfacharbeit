\iffalse
Einteilung in: Online-Marktplätze + Online-Händler, Intermediäre, Kataloversender, stationäre Händler, Hersteller/Marken \cite{Graf}

entwicklung des kaufprozesses: graf:abbildung; 

https://edoc.sub.uni-hamburg.de/hcu/volltexte/2017/370/pdf/Ebert_Kirsten.pdf Anfang

änderung kaufablauf: \cite{Schaefers}
\fi

% ALLGEMEIN
Bis Anfang 2020 war das Konsumverhaten in Deutschland im Allgemeinen geprägt durch eine starkte Kaufkraft, z. T. dank dem 0\%-igen Leitzins der \ac{EZB}\cite[S. 49]{Ebert} - jedoch ist die Kaufkraft durch die derzeit vorherrschende Corona-Situation abgeschwächt worden\cite{BfWE}. 

Außerdem haben sich die Bedürfnisse innerhab der letzten Jahrzente stark geändert: statt gleichbleibenden, rationalen Käufen und Kaufmotiven in den 1950ern, die die Auswahl der gekauften Güter stark abhängig von der zur Verfügung stehenden Geldmenge machten\cite[S. 38]{Schramm}; herrscht heute ein deutlich dynamischeres Kaufklima:
\begin{quote}
"So beziehen jetzt zum Beispiel auch solvente Kunden ihre Lebensmittel aus dem Billigdiscounter, während  umgekehrt  einkommensschwächere  Schichten  zu  Luxusgütern  greifen."\cite[S. 43]{Nitt}
\end{quote}

Im folgendem werde ich Verkäufer aller Art ähnlich wie im Buch "Das E-Commerce Buch: Marktanalysen - Geschäftsmodelle - Strategien" unterteilen: in Online-Marktplätze, Online-Händler, Kataloversender, stationäre Händler und Hersteller\cite[S. 18ff]{Graf}. Dabei sind Online-Marktplätze eine Art Online-Vermittler zwischen Kunden und Verkäfer, Online-Händler bieten dagegen nur eigene, meist sehr spezialisierten Sortimente an. Kataloversender verhalten sich ähnlich: sie versenden ihr Sortiment direkt an Kunden. Stationäre Händler verkaufen im Gegensatz zu den genannten in Fillialen und sind am stärksten von den Änderungen der letzten Jahrzenten betroffen. Während sich die genannten Unternehmensarten meistens am Ende der Verkaufskette befinden, stehen Händler oft am Anfang: sie stellen Güter her und sind dementsprechend, wenn sie nicht selber verkaufen, auf weitere Unternehmen für Verkauf und Vermarktung angewiesen(ebd.). %vor und -nachteile?


\iffalse 
    VERÄDERUNGEN bzgl der Verkäuferstruktur
    
        individiuelle Produkte
        
        sozialer Aspekt: Einzelhandel
        
        Einzelhandel: erst anbieter, dann produkt; onlinehandel umgekehrt
            .> weil einfacheres vergleichen, mehrere läden lohnen sich nicht
            einzelhandel weniger preissensibel; jedoch ist onlinehandel vor allem in der Elektronikbranche konkurrenzfähig, sh. Media Markt\cite[S. 27]{Graf}
            ABBILDUNG GABLER S. 313
        Mutichannel: Onlinebestellung im laden abholen/zurückgeben, wenig mehrwert (S 37) 
            
        {
        Katalogversender bieten ab 2002 online an: +56\% Umsatz, S. 30f
        }
        
        Hersteller: "Powerseller" kaufen produkte von herstellern, verkaufen siedeutlich unter marktpreisen weiter -> probleme mit preisverfall S. 31
        schrecken aufgrund von biosherigen vertreib über fachhandel meist davor zurück, online-marktplätze einzurichten, vereinzelt(hugoboss.com) S. 34
        
        allgemein online: großes wachstum ab 2002 S. 32
            jüngere kaufen mehr online ein S 37f
            
    STEHENGEBLIEBEN: S 39
\fi

Zusätzlich hat sich unter vielen Konsumenten das Bedürfnis nach individiuellen und auf den Käufer angepassten Produkten gebildet\cite[S. 43]{Nitt}, was wahrscheinlich durch die extrem große Außwahl bei dem Online-Shopping hervorgerufen wurde. In diesem Aspekt kann der stationäre Einzelhandel schlicht nicht mithalten, da Raum für Produkte begrenzt und sehr teuer ist. 
\iffalse
> billig - strategie vorallem von amazon und alibaba

veranschaulichung eigene tabelle börsenwert einzelner unternehmen

Dabei gibt es auch unternehmen, die mehrere berieche nutzen, wie amazon
\fi

Allerdings hat der stationäre Handel noch einen bedeutenden Vorteil: den sozialen Aspekt. Dieser wird vermutlich mit fortschreitender Digitalisierung eine immer wichtigere Rolle spielen\cite[S. 50]{Ebert}.
\begin{quote}
"Als Mittel gegen Vereinsamung und Anonymisierung im Alltag wird die soziale Komponente beim Einkaufen [...] zunehmend an Bedeutung gewinnen."\cite[S. 43]{Nitt}
\end{quote}
