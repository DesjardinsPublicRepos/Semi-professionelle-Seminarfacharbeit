%zusammenfassung

%gegenseitige beeinflussung: showrooming-effekt: kauf online aufgrund von stationärere beratung -- kauf stationär nach onlinerecherche \cite[S. 21f]{evilcom}

%wie viele läden haben in schleusingen einen onlineshop?


\begin{folding} %stationärer Handel

Aufgrund der Verschiebung von Nachfrage und Bedürfnissen von Konsumenten in den letzten Jahrzehnten wird der stationäre Handel trotz Multichannel-Versuchen keine einfache Zukunft haben. So kann in vielen Fällen stattdessen direkt von Herstellern gekauft werden, die mittlerweile den Vertriebswegwechsel von \ac{B2B} zu \ac{B2C} weitesgehend hinter sich haben. 
Insbesondere in Thüringen steht es im Vergelich zum Rest Deutschlands dank der Kombination aus schlechter Kaufkraft und niedriger Bevölkerungszahl pro Fläche schlecht um den konventionellen Einzelhandel\cite[S. 29]{Nitt}. Dazu kommen demografische Änderungen, die insbesondere in der Mitte Deutschlands Probleme verursachen: so nimmt die Bevölkerung z. B. in Hamburg, trotz Schrumpfen der Bevölkerungszahl, zu - jedoch nicht in Hildburghausen, einer der Landkreise, die am meisten Bewohner verliert\cite[S. 32f]{Nitt}. 
Zum Glück einiger Vertriebe treffen diese schlechten Chancen nicht auf alle Branchen zu - der Lebensmittelvertrieb hat z. B. kaum Online-Konkurrenz[Umfrage]. Um in den restlichen Geschäftssektoren einen maximale großen Umsatz zu erzielen, sollte der konventionelle Einzelhandel aufgrund der alternden Bevölkerung, die meist noch stationär kauft, vorerst Investitionen für die wachsende Gruppe von Senioren und dementsprechend Erreichbarkeit o. ä. nutzen.

\end{folding}

\begin{folding} %Umweltverschmutzung

Zudem kommt immer öfter das Argument auf, dass der Wechsel zum Onlinehandel umweltschädlich sei, da mehr Lieferfahrzeuge unterwegs sind. Jedoch haben bereits die Autoren des "Evil Commerce [...]"-Buches diese These anhand einer Modellrechnung weitesgehend wiederlegt. Sie berechneten eine 90\%-ige Kraftstoffersparniss bei komplettem Umstieg zum Distanzhandel in Großstädten unter optimalen Bedingungen\cite[S. 25f]{evilcom}. Um genauerer Aussagen bezüglich des ländlichen Raumes zu treffen, werde ich die genannte Rechnung bzgl. Entfernung - da Einkaufszentren nicht berücksichtigt wurden - modifizieren und insofern erweitern, dass zusätzlich ein Einkauf in mehreren Geschäften nacheinander mit in Betracht gezogen wird.

\begin{itemize}

\item Im meiner Modellrechnung kaufen 100 Bewohner eines Dorfes in einer 4km entfernten Stadt ein. Sie kaufen im Durchschnitt in 3 von 10 Einkaufsmöglichkeiten ein, die je 500m voneinander entfernt sind. Dabei gehe ich davon aus, dass alle Kunden über eine 500m lange Straße innerhalb des Dorfes zu erreichen sind.

\item Wenn alle Bewohner stationär kaufen, legen sie im Durchschnitt eine Strecke von 

\begin{align}(250m + 4000m + 3 \cdot 500m + 4000m + 250m) \cdot 100 = 1000000m\end{align}

 zurück. Dabei nehme ich an, dass alle Bewohner denselben Ortsausgang benutzen und somit einen durchschnittlichen Weg von 250m zu diesem besitzen.

\item Wenn jeder Verkäufer jedoch die Güter an seine im Durchschnitt 30 Kunden versendet, müssten alle Lieferwagen zusammen eine Strecke von gerade einmal 

\begin{align}(4000m + 500m + 4000m) \cdot 10 = 85000m\end{align}

zurücklegen.
\item In dieser Darstellung hat der Distanzhandel eine ähnlich hohe Kraftstoffersparniss - 91.5\%. 

\end{itemize}
Zwar kann mithilfe dieses Modelles die These der Umweltverschmutzung auch auf dem Land im Allgemeinen wiederlegt werden, jedoch ist sie keine genaue Darstsellung der Realität, da viele Faktoren, wie z. B. die Retouranzahl, die beispielsweise in der Modebranche überproportional hoch ist, nicht beachtet wurden(ebd.).

\end{folding}
