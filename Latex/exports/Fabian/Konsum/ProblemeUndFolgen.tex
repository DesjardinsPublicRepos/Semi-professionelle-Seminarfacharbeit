%zusammenfassung

%hersteller b2b zu b2c

%gegenseitige beeinflussung: showrooming-effekt: kauf online aufgrund von stationärere beratung -- kauf stationär nach onlinerecherche \cite[S. 21f]{evilcom}




\begin{folding} %Umweltverschmutzung

Zudem kommt immer öfter das Argument auf, dass vermehrtes Einkaufen Umweltschädlich sei, da mehr Lieferfahrzeuge unterwegs sind. Jedoch haben bereits die Autoren des "Evil Commerce [...]"-Buches diese These anhand einer Modellrechnung weitesgehend wiederlegt. Sie berechneten eine 90\%-ige Kraftstoffersparniss bei Kompletten Umstieg zum Distanzhandel in Großstädten unter optimalen Bedingungen\cite[S. 25f]{evilcom}. Um genauerer Aussagen bezüglich des ländlichen Raumes zu treffen, werde ich die genannte Rechnung bzgl. Entfernung - da Einkaufszentren nicht in Betracht gezogen wurden - modifizieren und insofern erweitern, dass zusätzlich ein Einkauf in mehreren Geschäften nacheinander mit in Betracht gezogen wird.

\begin{itemize}
\item Im meiner Modellrechnung kaufen 100 Bewohner eines Dorfes in einer 4km entfernten Stadt ein. Sie kaufen im Durchschnitt in 3 von 10 Einkaufsmöglichkeiten ein, die je 500m voneinander entfernt sind. Dabei gehe ich davon aus, dass alle Kunden über eine 500m lange Straße innerhalb des Dorfes zu erreichen sind.

\item Wenn alle Bewohner stationär kaufen, legen sie im Durchschnitt eine Strecke von (4000m + 3 * 500m + 4000m) * 100 = 950000m zurück.latex

\item Wenn jeder Verkäufer jedoch die Güter an seine im Durchschnitt 0.3 * 100 = 30 Kunden versendet, müssten alle Lieferwagen zusammen eine Strecke von gerade einmal (4000m + 500m + 4000m) * 10 = 85000m zurücklegen.

\item In dieser Darstellung hat der Distanzhandel eine ähnlich hohe Kraftstoffersparniss - 89\%. 
\end{itemize}
Zwar kann mithilfe dieses Modelles die These der Umweltverschmutzung auch auf dem Land wiederlegt werden, jedoch ist sie keine genaue Darstsellung der Realität, da viele Faktoren, wie z. B. die überproportional hohen Retourraten der Modebranche(ebd.).

\end{folding}
