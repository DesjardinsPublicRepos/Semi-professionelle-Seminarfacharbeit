Als Amazon, anfangs noch \emph{cadabra.com}, am 5. Juli 1994 von Jeff Bezos und seiner Frau McKenzie gegründet wurde, hatte wahrscheinlich niemand die Vision eines marktführendem Online-Unternehmens im Kopf - im Gegensatz, Amazon war ursprünglich ein Online-Buchhandel für bestimmte, seltene Bücher\cite[S. 17]{Graf}. Trotz der kleinen Zielgruppe wuchs das Unternehmen in den folgenden Jahren bedeutend: schon zwei Jahre später wurden Aktien angeboten, außerdem wurde anfangs noch fast der komplette Gewinn reinvestiert\cite{Rosoff}, was das Aufkaufen ganzer Unternehmen schon 4 Jahre nach der Gründung ermöglichte, bespielsweise von \emph{pets.com} und \emph{overstock.com}\cite{ChannelAdvisor}. An der Strategie, Unternehmen komplett zu kaufen, hat sich bis heute nichts geändert - im Gegenteil, Amazon kauft heute mehr und größere Unternehmen als je zuvor\cite[S. 27]{Haendlerbund}, wie Audible, Kiva Systems oder Twitch\cite{Sherman}. Der Gewinn wird jedoch nicht mehr ausschließich reinvestiert\cite{Rosoff}. Mit der Zeit expandierte die Firma in viele weitere Gebiete: Cloud Computing mit \ac{AWS} 2002 sowie Musik mit einem Online-Musik-Store und Lebensmittel mit AmazonFresh im Jahr 2007\cite{Sherman, ChannelAdvisor}. Auch bezüglich des Onlinehandels breitete Amazon ab 2000 nach und nach die Produktauswahl aus, wodurch sich der darmalige Buchhandel zu dem heutigen Onlineversandhandel für fast alle Produktereiche entwickelte. Ein wichtiger Schritt zu diesem Ziel war das Ermöglichen von Drittanbieter-Verkäufen ab dem 30. September 1999, was die Bekanntheit und Anzahl der Verkäufe erheblich steigerte\cite{Sherman}. Außerdem wurden weitere Technologien wie Amazon Prime und AmazonBasics entwickelt, die den Onlinehandel und -versand unterstützen\cite{ChannelAdvisor}, aber auch alleinstehende Projekte, wie Kindles, das Fire Phone oder Smart-Home-Geräte\cite{Sherman}.

Amazon’s Haupteinnahmequelle ist mit 84\% der Einnahmen zweifelsfrei der Onlinehandel und -versand\cite[Abb. 5]{Desjardins} - und genau dieser hat in den letzten Jahren einige Probleme hervorgerufen. Bespielsweise führt die Verkauffstrategie Amazons, Produkte so billig wie möglich und mit kostenlosem Versand anzubieten, um mehr Käufer anzusprechen\cite{Quartz}, zum Einsparen von Ausgaben in fast allen Gebieten - auch im Bezug auf Arbeiter\cite[S. 6]{Apicella}. So werden insbesondere in der Weihnachtszeit Leiharbeiter eingestellt. In der ARD-Reportage "Ausgeliefert! Leiharbeiter bei Amazon" wird 2013 gezeigt, wie deren Arbeitsalltag aussah: Zu siebt wird in einer Ferienwohnung übernachtet, oft bekommen die Angestellten nur wenige Stunden Schlaf. Jeden Tag aufs neue ist es unsicher, ob man gebraucht wird - wenn nicht, gibt es keinen Lohn. Mitarbeiter der Dienstleistungsgewerkschaft Ver.di und Amazons erklären, dass 2013 in Koblenz circa 3100 von 3300 Arbeitern befristet angestellt waren\cite{Ausgeliefert}.
Außerdem existiert ein hoher Grad an Überwachung und Kontrollen, wie Apicella in ihrer Studie andhand der Stadt Leipzig beschreibt\cite[S. 29]{Apicella}:
\begin{quote}
"Die Verkaufsarbeit durchläuft dabei einen Prozess der [...] vollständige[n] Überwachung und Disziplinierung der Beschäftigten[...]."
\end{quote}
Dementsprechend sind Streiks bei Amazon keine Seltenheit: Beispielsweise streikten Angestellte in Deutschland zwei Monate nach der besagten Reportage unter dem Motto "Wir sind keine Roboter" gegen niedrige Löhne, befristete und schlechte Arbeitsverhältnisse sowie die starke Digitalisierung der Arbeit\cite[S. 6]{Apicella}. Amazon reagierte in den folgenden Jahren mit mehreren Lohnerhöhungen, jedoch exestieren noch vereinzelt Streiks\cite{JGraf}.

Da in Folge der Corona-Krise im 1. Quartal von 2020 die Verkäufe um 32\% stiegen, bekamen Angestellte eine weitere Lohnerhöhung. Zusätzlich wurden 175000 neue Stellen ausgeschrieben - nicht nur, weil mehr Arbeiter gebraucht werden, sondern auch weil einige Angestellte aufgrund von "unsicheren Bedingungen" zu Hause geblieben sind bzw. dies immer noch tun. Zusätzlich hat die Regierung Frankreichs aus demselben Grund 6 Einrichtungen geschlossen, obwohl Amazon beteuerte, dass genügend Sicherheitsmaßnahmen vorhanden waren. Bezos sagte außerdem, dass, um die Logistik zu erhalten, die Sicherheit der Arbeiter zu gewährleisten und Überstunden zu bezahlen, fast der Gesamte Gewinn reinvestiert werden wird\cite{Theweek}.

Innerhalb der letzen 26 Jahre hat Amazon sich von einem Online-Buchhandel zu einem weltweiten Onlinehändler fast alle Produktklassen entwickelt. Außerdem bietet die Firma auch andere Dienste an, wie z. B. Cloud Computing mit \ac{AWS}. Jedoch steht das Unternehmen bezüglich der Arbeitsbedingungen seit fast einem Jahrzehnt in der Kritik.
