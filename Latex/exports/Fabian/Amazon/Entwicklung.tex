Als Amazon, anfangs noch \emph{cadabra.com}, am 5. Juli 1994 von Jeff Bezos und seiner Frau McKenzie gegründet wurde, hatte wahrscheinlich niemand die Vision eines marktführendem Online-Unternehmens im Kopf - im Gegensatz, Amazon war ursprünglich ein Online-Buchhandel für bestimmte, seltene Bücher\cite[S. 17]{Graf}. Trotz der kleinen Zielgruppe wuchs das Unternehmen in den folgenden Jahren bedeutend: schon zwei Jahre später wurden Aktien angeboten, außerdem wurde anfangs noch fast der komplette Gewinn reinvestiert\cite{Rosoff}, was das Aufkaufen ganzer Unternehmen schon 4 Jahre nach der Gründung ermöglichte, von bespielsweise \emph{pets.com} und \emph{overstock.com}\cite{ChannelAdvisor}. An der Strategie, Unternehmen komplett zu kaufen, hat sich bis heute nichts geändert - im Gegenteil, Amazon kauft heute mehr und größere Unternehmen als je zuvor\cite[S. 27]{Haendlerbund}, wie Audible, Kiva Systems oder Twitch\cite{Sherman}. Der Gewinn wird jedoch nicht mehr ausschließich reinvestiert\cite{Rosoff}. Mit der Zeit expandierte die Firma in viele weitere Gebiete: Cloud Computing mit \ac{AWS} 2002 sowie Musik mit einem Online-Musik-Store und Lebensmittel mit AmazonFresh im Jahr 2007\cite{Sherman, ChannelAdvisor}. Auch bezüglich des Onlinehandels breitete Amazon ab 2000 nach und nach die Produktauswahl aus, wodurch sich der darmalige Buchhandel zu dem heutigen Onlineversandhandel für fast alle Produktereiche entwickelte. Ein wichtiger Schritt zu diesem Ziel war das Ermöglichen von Drittanbieter-Verkäufen ab dem 30. September 1999, was die Bekanntheit und Anzahl der Verkäufe erheblich steigerte\cite{Sherman}. Außerdem wurden weitere Technologien wie Amazon Prime und AmazonBasics entwickelt, die den Onlinehandel und -versand unterstützen\cite{ChannelAdvisor}, aber auch alleinstehende Projekte, wie Kindles, das Fire Phone oder Smart-Home-Geräte\cite{Sherman}.

Amazon’s Haupteinnahmequelle ist mit 84\% der Einnahmen zweifelsfrei der Onlinehandel und -versand\cite[Abb. 5]{Desjardins} - und genau dieser hat in der heutigen Zeit einige Probleme hervorgerufen. Bespielsweise sind von dem in [4.2] beschriebenen Ziel, alles so billig wie möglich zu halten, auch Angestellte, insbesonderre Leiharbeiter, betroffen\cite[S. 6]{Apicella}. Außerdem sind die Arbeitsbedingungen vieler Stellen Amazons bedeutend schlecht, wie Apicella in ihrer Studie andhand der Stadt Leipzig beschreibt\cite[S. 29]{Apicella}:
\begin{quote}
"Die Verkaufsarbeit durchläuft dabei einen Prozess der [...] vollständige[n] Überwachung und Disziplinierung der Beschäftigten[...]."
\end{quote}
Dementsprechend sind Streiks bei Amazon keine Seltenheit: Beispielsweise streikten deutsche Angestellte zwei Monate nach der ARD-Reportage "Ausgeliefert! Leiharbeiter bei Amazon"\cite{Ausgeliefert} unter dem Motto "Wir sind keine Roboter" gegen niedrige Löhne, befristete und schlechte Arbeitsverhältnisse sowie die starke Digitalisierung der Arbeit\cite[S. 6]{Apicella}.



\iffalse

dynamic pricing https://www.exeo-consulting.com/pdf/exeo_Dynamic%20Pricing%20Amazon_2016.pdf

Arbeitsbedingungen https://www.rosalux.de/fileadmin/rls_uploads/pdfs/Studien/Studien_09-16_Amazon_Leipzig.pdf


  >billg-wettbewerb: amazon macht tlw. verlust, um andere zu verdrängen, zB Prime Versand an einem Tag
  >verbilligung: Lohn und Arbeitsbedingungen schlecht "verbilligung nicht nur auf kosten des Gewinns, sondern auch der Arbeiter"



>CORONA, besordere Artikel wie Toilettenpapier oder desinfektionsmittel

>WEGSCHMEI?EN ABGELAUFENER LEBENSMITTEL > onlinehandel wenoger sclimm weil spielraum
-> TAUSCH REINFFOLGE   AMAZON - AUSWIRKUNGEN ONLINEHANDEL AUF INFRASTRUKTUR
DIE IN 5.1 BESCHRIEBENE AUSBEUTUNG UND VERBILLIGUNG WURDE GRÖ?TENTEILS DURCH AMZON HERVORGERUFEN

>Ausnutzung
>Kontakt mit Amazon

>Schlussfolgerungen: auf Tühringen übertragbar, weil 

9.3 Schlussfolgerungen*/
\fi
