

\iffalse
 alles einfacher und unkompliziert

 Vorreiter in sachen niedrige Preise > ist sehr wichtig, weil
   viel einfacher vergelichbar, qualität des Produkts nicht einfach einsehbar: sie muss nicht außergewöhnlich, nur akzeptabel sein - jedoch auch nicht schlecht, da 14-tage-rückgabe ohne angabe eines grundes

 einfluss extrem in coronazeiten

 S 49 https://edoc.sub.uni-hamburg.de/hcu/volltexte/2017/370/pdf/Ebert_Kirsten.pdf
 danach: modell für veränderung
\fi


Bis Anfang 2020 war das Konsumverhaten in Deutschland für Online- und Einzelhandel geprägt durch eine starkte Kaufkraft - jedoch ist diese durch die derzeit vorherrschende Corona-Situation abgeschwächt worden\cite{BfWE}. Allerdings wird die Nachfrage durch den 0\%-igen Leitzins der \ac{EZB} wieder erhöht, da Sparen so relativ unattraktiv ist\cite[S. 49]{Ebert}.

Außerdem haben sich die Bedürfnisse innerhab der letzten Jahrzente stark geändert: statt gleichbleibenden, rationalen Käufen und Kaufmotiven in den 1950ern, die die Auswahl der gekauften Güter stark abhängig von der zur Verfügung stehenden Geldmenge machten\cite[S. 38]{Schramm}, herrscht heute ein deutlich dynamischeres Kaufklima:
\begin{quote}
"So beziehen jetzt zum Beispiel auch solvente Kunden ihre Lebensmittel aus dem Billigdiscounter, während  umgekehrt  einkommensschwächere  Schichten  zu  Luxusgütern  greifen."\cite[S. 43]{Nitt}
\end{quote}

Zusätzlich hat sich unter vielen Konsumenten das Bedürfnis nach individiuellen und auf den Käufer angepassten Produkten gebildet(ebd.), was wahrscheinlich durch die extrem große Außwahl bei dem Online-Shopping hervorgerufen wurde. In diesem Aspekt kann der stationäre Einzelhandel schlicht nicht mithalten, da Raum für Produkte begrenzt und sehr teuer ist.

Allerdings hat der stationäre Handel noch einen bedeutenden Vorteil: den sozialen Aspekt. Dieser wird vermutlich durch fortschreitende Digitalisierung eine wichtige Rolle spielen\cite[S. 50]{Ebert}.
\begin{quote}
"Als Mittel gegen Vereinsamung und Anonymisierung im Alltag wird die soziale Komponente beim Einkaufen [...] zunehmend an Bedeutung gewinnen."\cite[S. 43]{Nitt}
\end{quote}
