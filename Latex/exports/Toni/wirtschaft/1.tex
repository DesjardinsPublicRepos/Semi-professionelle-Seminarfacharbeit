Die Gesamteinnahmen des Onlinehandels hängt von den Einnahmen der Unternehmen untrennbar ab. Da in der Entwicklung des Onlinehandels schon auf die wirtschaftlichen Erfolge der letzten 20 Jahre eingegangen worden ist, wird dies eine Analyse über den momentanen Stand der Dinge, speziell beeinflusst durch die Pandemie.
Seit dem 7. Januar 2020, bei der Identifizierung in der chinesischen Stadt Wuhan des damals neuartigen COVID-19-Virus, versuchten die Länder die Krankheit einzudämmen. Die Folgen waren unvorhersehbar und diese haben nicht nur die Normalität, sondern auch die Wirtschaft erschüttert \cite{who}. Am 27. Januar ist bereits der erste Deutsche infiziert, drei Tage später ruft die WHO eine Gesundheitsnotlage internationaler Tragweite aus \cite{mdr-aktuell}. Bis zum 16. März sind nahezu alle Kitas und Schulen geschlossen, darunter auch unser Schleusinger Gymnasium. 6 Tage später kommt es zum kompletten Lockdown, was als einzige Möglichkeit zum Kauf eines nicht-essenziellen Produktes nur den Onlinehandel zulässt, da alle sonstigen lokalen Läden geschlossen haben. Dies hat zu horrenden Verlusten der Geschäfte geführt, besonders Restaurants, die keine Lieferung anbieten konnten, haben das Größte negativ verzeichnet. Auch der DAX hat ein Rekordverlust gemeldet, der nur mit dem des 11. September 2001 verglichen werden kann.
