Der Onlinehandel hat den Verkauf, speziell von kleineren Unternehmen, revolutioniert. Nicht nur durch Onlinemarketing, sondern auch durch den Verkauf auf Seiten anderer Anbieter konnten Unternehmen ihre Kundenzahl erhöhen und ihre Profite maximieren. Auch Produkte, die durch Handarbeit hergestellt werden, konnten ihre Position im Markt finden und sich meist als Luxusgut etablieren. Das Unternehmen hat sich dabei geschichtlich als größter Profiteur des Onlinehandels herausgestellt. Es konnte seine Profite durch das Vergrößern der Kundschaft maximieren und durch das bereits angesprochene Onlinemarketing begründen. Hingegen hat auch das ganze seine Folgen, die nicht zu vernachlässigen sind. Darunter fällt die Behandlung und Arbeitsverhältnisse der Arbeiter als auch die Umwelt und der Beitrag zum Klimawandel. 