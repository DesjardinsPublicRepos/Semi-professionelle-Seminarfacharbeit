PayPal ist eine Onlinebezahlmethode (Obm), die besonders durch das Aufstreben anderer Onlineversandhändler, wie zum Beispiel Amazon, an Aufmerksamkeit gewann. Laut PayPals eigenen Angaben besitzen knapp 325 mio. Personen ein aktives Konto beim Anbieter. Ein weiterer Vorteil gegenüber anderen Obm ist eine Vielzahl an Währungen in denen gezahlt werden kann. Auf den Stand März 2020 liegt diese Zahl bei 100 verschiedenen Währungen, während nur 56 ausgezahlt werden können und nur 25 verschiedene Währungen als Standartwährung für das PayPal-Konto benutzt werden können. (https://www.paypal.com/us/webapps/mpp/about).   
Es ist heut-zu-Tage einer der meist genutzten (https://de.statista.com/statistik/daten/studie/224827/umfrage/marktanteile-von-zahlungsverfahren-beim-online-handel/)  Obm in Deutschland und ist auch in unseren hildburghäuser Raum sehr weit verbreitet. Aus diesem Grund werde ich die Grundstrukturen, deren daraus entstandene Entwicklung und den darauf folgenden Einfluss analysieren und dementsprechend auswerten.


\cite[S. 30]{toni}