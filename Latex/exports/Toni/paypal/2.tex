<<<<<<< Updated upstream
Auch wenn PayPal seine Vorteile mit sich bringt hat es seine Nachteile. Speziell die weitere Nutzung der Daten, die PayPal generiert, ist fraglich. Dabei sendet PayPal Daten an das US-Heimatschutzministerium die die Daten abgleicht mit möglichen Straftätern oder Terroristen, dabei kann schon der geringste Verdacht zu einer Sperrung des Kontos führen. (https://www.it-recht-kanzlei.de/Brauchen_wir_PayPal.html) Ebenfalls ist das Einfrieren von Konten auch ein großer Nachteil, da eine Vielzahl von Menschen der Geldwäsche von PayPal bezichtigt werden, aber ohne jeglichen Beweis. In beiden Fällen kann man Wiederspruch einlegen, aber man muss Nachweißen das alle Zahlungen legal waren und sich ausweisen können. Bis zu dem Zeitpunkt der Wiedereröffnung des Kontos oder Wiederaktivierung falls der Wiederspruch angenommen worden ist können einige Monate vergehen und man hat absolut keinen Zugriff auf das Geld. Dazu hat man keinen Ansprechpartner in Deutschland für einen Beschwerde, sondern muss sich an die englische Behörde wenden, wodurch auch englisches Recht gilt, dadurch ist das Unternehmen nahezu unantastbar in Deutschland. 
=======




Durch diese Grundstrukturen hat sich ein Wirtschaftssektor gebildet der profitabelste Sektor für ein großes Unternehmen ist. Besonders durch die Einbeziehung des Onlinehandels konnte der E-Commerce erst seinen Aufstieg feiern. Jedoch ist das Potential des Sektors noch lange nicht ausgeschöpft. Durch die Entwicklung von Sprachassistenten und Künstlichen Intelligenzen wird die Zeit des Prozesses des Einkaufens bis zum minimalen verringert. Auch durch das Integrieren von Drohen wird die Allgegenwärtigkeit des Onlinehandels früher oder später offensichtlich. %(https://www.tagesschau.de/wirtschaft/amazon-drohnen-101.html) 
Die Steigerung des Kommerzes, ins besonderen in ländlichen Regionen ist überwältigend. Vom Jahr 2000 bis 2020 hat sich der Umsatz in Deutschland mehr als verfünfzigfacht. Von 1,3 Milliarden € auf 59,2 Milliarden € in nur zwanzig Jahren. %(https://de.statista.com/statistik/daten/studie/3979/umfrage/e-commerce-umsatz-in-deutschland-seit-1999/)
Auch zeigen die tendenzen in den kommenden Jahren nur nach oben. Bis 2023 sollen 71,4 Milliarden € Umsatz im Sektor E-Commerce gemacht werden.       
 
%(https://parcellab.com/wp-content/uploads/2019/09/statistic_id488012_prognose-der-e-commerce-nutzer-in-deutschland-bis-2023.png) 
>>>>>>> Stashed changes
