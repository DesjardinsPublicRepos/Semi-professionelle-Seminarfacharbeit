



Durch diese Grundstrukturen hat sich ein Wirtschaftssektor gebildet der profitabelste Sektor für ein großes Unternehmen ist. Besonders durch die Einbeziehung des Onlinehandels konnte der E-Commerce erst seinen Aufstieg feiern. Jedoch ist das Potential des Sektors noch lange nicht ausgeschöpft. Durch die Entwicklung von Sprachassistenten und Künstlichen Intelligenzen wird die Zeit des Prozesses des Einkaufens bis zum minimalen verringert. Auch durch das Integrieren von Drohen wird die Allgegenwärtigkeit des Onlinehandels früher oder später offensichtlich. (https://www.tagesschau.de/wirtschaft/amazon-drohnen-101.html) 
Die Steigerung des Kommerzes, ins besonderen in ländlichen Regionen ist überwältigend. Vom Jahr 2000 bis 2020 hat sich der Umsatz in Deutschland mehr als verfünfzigfacht. Von 1,3 Milliarden € auf 59,2 Milliarden € in nur zwanzig Jahren. (https://de.statista.com/statistik/daten/studie/3979/umfrage/e-commerce-umsatz-in-deutschland-seit-1999/) Auch zeigen die tendenzen in den kommenden Jahren nur nach oben. Bis 2023 sollen 71,4 Milliarden € Umsatz im Sektor E-Commerce gemacht werden.       
 
(https://parcellab.com/wp-content/uploads/2019/09/statistic_id488012_prognose-der-e-commerce-nutzer-in-deutschland-bis-2023.png) 
