Auch wenn PayPal seine Vorteile mit sich bringt hat es seine Nachteile. Speziell die weitere Nutzung der Daten, die PayPal generiert, ist fraglich. Dabei sendet PayPal Daten an das US-Heimatschutzministerium die die Daten abgleicht mit möglichen Straftätern oder Terroristen, dabei kann schon der geringste Verdacht zu einer Sperrung des Kontos führen. (https://www.it-recht-kanzlei.de/Brauchen_wir_PayPal.html) Ebenfalls ist das Einfrieren von Konten auch ein großer Nachteil, da eine Vielzahl von Menschen der Geldwäsche von PayPal bezichtigt werden, aber ohne jeglichen Beweis. In beiden Fällen kann man Wiederspruch einlegen, aber man muss Nachweißen das alle Zahlungen legal waren und sich ausweisen können. Bis zu dem Zeitpunkt der Wiedereröffnung des Kontos oder Wiederaktivierung falls der Wiederspruch angenommen worden ist können einige Monate vergehen und man hat absolut keinen Zugriff auf das Geld. Dazu hat man keinen Ansprechpartner in Deutschland für einen Beschwerde, sondern muss sich an die englische Behörde wenden, wodurch auch englisches Recht gilt, dadurch ist das Unternehmen nahezu unantastbar in Deutschland. 