 
\setspeaker{Andre}[Prof. Spindler]
\setspeaker{Fabian}[Fabian Beez]
\setspeaker{Toni}[Toni Hausdörfer]
\addtolength{\transcriptlen}{1.5em}

\begin{description}

\Fabian Wir haben ja schon voerherein herausgefunden, dass Sie eine eigene Website haben. Dieses Interview führen wir zu einem Großen Teil unter dem Aspekt Onlinehandel bzw. -präsentation - deshalb spielt ihre Wesite aus unserer Sicht eine bedeutende Rolle. Deshalb haben wir auch ein paar Fragen in diese Richtung. Erst einmal ganz allgemein Wann haben und ist es in Europa? Seit wann ist es in Ihrer Branche üblich, sich online zu präsentieren?

\Andre In der Branche ist es eigentlich seit Anfang an üblich – also seit dem es dieses Medium gibt, weil die Architekten natürlich im besonderen Maße visuell werben, also mit ihrem Werk zeigen, was sie gebaut haben, sie zeigen vielleicht auch die einzelnen Phasen der Planung und der Fertigstellung. Mit Fotos, mit Videos, mit Drohnenüberflügen und ähnlichem. Das wirbt quasi für die Qualität des Architekten und zeigt auch dem potentiellem Bauherrn, was schon in welcher Qualität geleistet wurde. Das Internet ist also ein ganz wichtiges Medium. Ich weiß auch, dass vorallem die großen Büros viel Wert darauf legen, aktuelle Projekte vorzustellen und sich damit interessant zu machen.

\Toni Würden Sie demnach sagen, dass die Onlinepräsentation wesentlich wichtiger ist als die Eigenpräsentation durch Printmedien?

\Andre Ja, das denke ich schon. Das hat extrem zugenommen, denn bevor es Onlinepräsentationen gab, war eigentlich nur die Möglichkeit der Veröffentlichung in Zeitschriften an den potentiellen Bauherrn heranzutreten oder über das eigentlich gebaute Werk, also jemand der ein Haus gebaut hat, der hatte Bekannte die auch ein Haus bauen wollten. Da hat man den gefragt, wer war denn da Architekt? Wer hat denn für dich geplant? Und wenn man zufrieden war mit dem Architekten, dann hat man den empfohlen. Und das ist schwierig, mit einer solchen Art und Weise der Werbung, zu überleben. 
Außerdem werden auch Wettbewerbe ausgeschrieben, außerdem auch online, mittlerweile in fast allen Fällen. Man beteiligt sich an einem Wettbewerb, und wenn man da gut ist, bekommt man einerseits für die Bearbeitung ein wenig Geld, aber wenn man den Auftrag tatsächlich bekommt, dann ist das auch sehr wirtschaftlich. Aber es kann eben bei 50 Teilnehmern nur einer gewinnen – also ist die Chance nicht sehr hoch.

\Fabian Sie hatten ja auch vorhin erwähnt, dass Sie Bauprojekte online stellen, um diese zu präsentieren. Ich habe jetzt auf Ihrer Webseite leider keins gefunden. Wo würde das denn üblich hochgeladen werden?

\Andre Ja, wir sind kein gutes Beispiel für eine gepflegte Website. Ich will das jetzt auch nicht groß entschuldigen, weil für uns diese Dinge, die ich eben gerade über Architekten gesagt habe, nicht ganz zutreffen. Wir bilden ja so eine Teilleistung des Architekten oder Ingenieurs Daseins ab - Wir beraten, wir haben spezielle Aufgaben bei der Planung am Bau und die lassen sich nicht so gut darstellen wie das fertige Werk. Wir würden, das ist immer noch so in der Planung und teilweise in der Vorbereitung spezielle Planung vorstellen. Natürlich, die fertigen Häuser auch. Aber wir werben ja nicht mit einer schönen Fassade wie der Architekt, sondern wir werben mit einer Fachplanung, die man möglichst gar nicht sehen soll, sondern die dann wirkt, wenn es brennt und die vorher gar nicht in Erscheinung treten soll. Sie alle kennen sicherlich die die Feuerlöscher, die grünen Männchen über den Türen und solche Dinge, die man eben wahrnimmt. Wir machen natürlich viel mehr in Konzeption, die sich nicht so leicht visualisieren lassen. Also das ist die fachliche Ausrede für das, was wir nicht leisten. Und eine andere Ausrede ist einfach, dass wir ganz wenig Zeit haben und dass wir zwar viel leisten auf diesem Gebiet, aber uns nicht gut verkaufen. Und auch das hat wieder den Grund, dass wir eher durch Empfehlungen - also gar nicht so sehr über online, sondern durch unsere Leistung seit Jahren immer wieder Kundschaft bekommen, vor allen Dingen Architekten, spezielle Bauherrengruppen, auch größere institutionelle Bauherren, die sagen, Ihr habt das gut gemacht, seit Jahren, wir würden euch gern wieder mit fürs nächste Projekt nehmen. Die schauen eben nicht ins Netz, weil die sagen naja, die haben jetzt ein neues Bildchen. Aber eigentlich wollen wir ja die Leistung haben.

\Fabian Also würden Sie sagen, dass mehr Kunden analog als online auf Sie stoßen?

\Andre In unserem Fall ja. Wir kriegen natürlich übers Netz Projektphasen zugesandt. Also Vorplanung, Ideen. Das ist natürlich heute üblich über online und nicht als Papierstapel. Und wir geben Angebote natürlich übers Internet ab. Alles digital. Wir bearbeiten. Jeder hat also mindestens einen Computer an seinem Arbeitsplatz, mehrere Bildschirme. Wir arbeiten nur digital. Aber das Herantreten, das erfolgt, wenn man so will, Analog.

\Toni Gibt es einen grossen Unterschied im Vergleich, als Sie eine Webseite gemacht haben bezüglich der Nachfrage vorher und danach?

\Andre Es gibt einen Unterschied, aber der ist nicht signifikant für die wirtschaftliche Ausbeute, um das mal so zu sagen. Also es gibt eine Reihe von Anfragen über die Website, weil wir dort auch ein Kontaktfeld haben und wir schätzen aber ein, dass das eher zufällige Dinge sind. Wenn das Wort Architect googelt findet man uns natürlich auch und hat aber, wenn man näher hinschaut, dann eigentlich nicht den Architekten, der in Einfamilienhäusern plant und baut oder eine kleine Sanierung macht. Da haben wir auch Anfragen, die sich speziell auf meine Person beziehen, weil ich eben eine Menge Erfahrung auf dem Gebiet habe, im Denkmalbereich und in Altbausanierung. Aber die meisten speziellen Anfragen zum Brandschutz, zur Planung, die kommen also unabhängig von der Website.

\Toni Noch eine Frage war beispielsweise, wie der Online-Aspekt, alse diese Webseiten und die mediale Repräsentation, wie die sich auswirken können, vielleicht auf ihr berufliches Umfeld und auch auf die konkurrierenden Firmen.

\Andre Also wir nutzen unsere Homepage nicht nur für zu Werbung oder Verkauf, in Anführungsstrichen. Wir verkaufen ja Wissen und das können wir nur sagen, dass dies der Fall ist. Aber wir verkaufen es nicht online wie Online-Handel. Wo wir gute Erfahrungen gemacht haben, ist die Nutzung dieser Online-Möglichkeit, bei der Information von nachgelagerten Personen. Das will ich kurz erläutern: Wir haben also ein Teil unserer Website mit Formblättern, mit Beantwortung von Fragen dieses "FAQ", z.B. was viele Firmen haben gefüllt. Das heißt, ich verlange in meiner Prüftätigkeit am Abschluss eines Bauvorhabens bestimmte Unterlagen, Nachweise, Unterschriften, Fotos und ähnliches. Das haben wir alles auf unserer Homepage dargestellt. In welcher Form wir das wollen; was, was wir unter bestimmten Begriffen verstehen, unter bestimmten Nachweisen fordern müssen. Und der Bauleiter, der Architekt, auch der Bauherr kann dort nachschauen und kann sich diese Informationen online herunterziehen und ist damit natürlich nicht mehr eine Belastung für uns. Wir müssen nicht alles am Telefon erklären oder ihn einladen, dass wir ihn schulen. Sondern er kann dieses Medium nutzen und sich dort die Form Blätter zum Ausfüllen herunterladen, damit arbeiten. Das halte ich für einen ganz großen Gewinn. Das wird auch reflektiert von den Baustellen, dass man sagt ja, das ist gut. Wir können doch abends um 10 uns etwas runterziehen, das durchlesen und morgen um 8 verlangen wir diese Dinge auf der Baustelle. Zu den Mitbewerbern war so ein bischen eine Frage. Da muss ich sagen, auf dem Niveau, auf dem wir arbeiten, haben das viele. Ich denke, die Mehrzahl der Büros unterstützen damit die ihnen nachgelagerten Baustellen.

\Fabian Noch eine Frage, die Sie eigentlich schon beantwortet haben - Spielt die Onlinepräsenz in ihrer Branche eine bedeutende Rolle? %(doppelt?)

\Andre Bei Architekten ja, bei unserem Spezialgebiet eher weniger.

\Fabian Welche Rolle spielt ihre Website in dem Konsumverlauf – Werden Kunden über die Seite auf Sie Aufmerksam oder hat sie eher eine Informationsfunktion?

\Andre Ich denke, es ist mehr die Informationsfunktion. Natürlich gibt es auch zufällige Kunden, die das Wort Architektur oder Brandschutz eingeben, auf uns stoßen. [...] Unsere Webite enthält die technischen Informationen und die Gesichter der Mitarbeiter, was ich übrigens auch für wichtig halte – mit wem man vielleicht ein Jahr zu tun haben wird, oder wer immer wieder die Fragen beantwortet oder die Unterlagen anfordert, dass man da mal ein Gesicht sieht – diese menschliche Basis. Dass man unsere Leistung mehr konsumiert, dass kann man nicht erwarten. Die Leute kommen ja nicht zu uns, um eine Handtasche zu kaufen, sondern um eine spezielle Dienstleistung von uns erledigen zu lasssen. Dort sind wir sicherlich auch immer wieder im Wettstreit mit Mitbewerbern, die das ählich anbieten, vielleicht auch einmal etwas kostengünstiger, oder die andere Vorteile haben, weil sie etwa in der Nähe des Vorhabens ihren Bürositz haben und damit schneller verfügbar sind, oder ähnliche Dinge.

\Fabian Inwiefern ist die Menschliche Komponente im Onlinebereich, die Sie soeben angesprochen haben, wichtig für Sie?

\Andre Ich habe ja schon erwähnt, dass die Mitarbeiter dort vorgestellt werden, natürlich mit deren Einverständnis - Wir filmen jetzt niemand heimlich und stellen ihn ins Netz. Wir hatten da eine Fotografin, die auch darauf geachtet hat, dass wir da gut rüber kommen und uns ordentlich präsentieren. Auf der Website wird ganz kurz eingeschätzt, natürlich neben dem Namen, welche Qualifikationen der Mitarbeiter oder die Mitarbeiterin haben, wofür sie zuständig sind, und das ist auch sehr geschickt gemacht. Man sieht immer das gesamte Team, und wenn man auf den Mitarbeiter mit der Maus geht, wird er farbig und seine Daten erscheinen. Ich halte das für sehr angenehm und das führt schon dazu, dass man ein persöhnlicheres Verhältnis zueinander hat. Ich sehe das auch in meiner Seminartätigkeit, mit der ich in Deutschland unterwegs bin. Die meisten kennen mich, weil man mich irgendwann mal gegoogelt hat und dan weiß, wie ich aussehe. Dann begrüßt man mich ganz freundlich, dann sage ich, ich kenne Sie ja leider gar nicht, weil mir das Gesicht eben nicht eingängig ist, aber manche kennen mich weil ich schon einmal irgendwo bekannt geworden bin, über solche Medien.

\Fabian Wir haben auch auf Ihrer Webseite gesehen, dass Sie Mustervorlagen für Erklärungen zur Verfügung stellen. Sie haben schon gesagt, das ist ein für das Bauwesen an sich. Haben diese doch einen anderen Nutzen?

\Andre Ja, manche Dinge sind ja auch nur, um aufklärend zu wirken. Also was steckt hinter bestimmten Begriffen? Ich sag mal ein Beispiel wenn ein Handwerker eine Feuerschutztür einbaut, dann wollen wir wissen von welcher Firma? Was kann dieses Produkt, diese Tür? Und wir wollen andererseits wissen hat es der Handwerker auch richtig eingebaut, hat er ja die Montageanleitung gelesen und verstanden und auch die richtigen Schrauben, die richtigen Dübel genommen. Und es wird kurz erläutert dort der erfahrene Handwerker wird zur sagen, ja, das kann ich schon, das weiß ich, das muss ich gar nicht lesen. Aber am Ende ist es schon gut, dass wir die gleiche Sprache sprechen und eben quasi auch ein, Ich will sie jetzt nicht Weiterbildung nennen, aber zumindest eine Konformität mit dem Wissen und Können der Handwerker erzeugen, sodass wir es am Ende leicht haben, die Leistung zu akzeptieren.

\Toni Wie hat denn die Modernisierung Einfluss darauf genommen, wie Sie beispielsweise Unternehmen wählen, die sie dann beauftragen?

\Andre Ja, Wir beauftragen selber nicht, sondern das machen dann die Architekten und die Bauherren, aber ich will ein anderes Beispiel erzählen, wo die Reise hingehen wird. Und da sind wir gerade dabei. Das wird also von Papier weitestgehend in der ganzen Genehmigungsphase, im Umgang mit den Bauämtern und den Prüfingenieuren wie mir, auf Elektronik umgestellt werden. Die nächsten zwei, drei Jahre. Das heißt, wir werden kein Papierbündel mehr bekommen zum Durchsehen und Abstempeln, sondern nur noch Dateien. Und das klingt so einfach. Ist es gar kein Problem Dateien zu versenden. Das machen wir ja seit vielen Jahren. Aber die Frage ist, wie werden die gespeichert? Über 30 oder 50 Jahre, weil es ja Dokumente sind. Wenn also in 20 Jahren ein Kindergarten brennt, wird man mich versuchen aufzusuchen mit meinem Wissen, mit meinen Dokumentationen. Es hat dort richtig gemacht worden? Und das muss man auch in elektronischer Form und auch der Austausch übers Internet sichern und handlebar machen. Da sind viele Fragen noch ungeklärt. Auch die Ämter, auch die zuständigen Ministerien wissen noch nicht genau, wo eben die Reise hingeht. Und es muss für jeden handhabbar sein, auch für den kleinen Planer auf dem Dorf, der vielleicht ein oder zwei Einfamilienhäuser macht. Auch der muss sich diesen elektronischen Gangarten unterwerfen. Und das wird spannend die nächsten Jahre. Ich mache zum Beispiel auf jede Zeichnung einen grünen Stempel und sage ja, die wurde von mir geprüft. Da unterschreibe ich den Stempel, trage eine Nummer ein und das Datum. Und ein Kollege hat mal so aus Spaß gesagt, wir können doch nicht auf dem Bildschirm stempeln. Wie machen wir das dann? Wie machen wir ein solches Dokument als geprüftes Dokument kenntlich? Wie sichern wir diese Prüfung, dass sie niemand fälschen kann? Sie wissen sicherlich besser sogar als ich, wie schnell man Elektronik auch missbrauchen kann. Und in Papierform ist das schwieriger möglich, ist auch möglich, aber in der Elektronik viel leichter. Und wie erfährt zum Beispiel auch der dänische Eisenflechter, der mit einem französischen Fahrer unterwegs ist, auf einer belgischen Baustelle? Welches ist jetzt der letzte Stand der Zeichnung? Was hat Spindler geprüft? Was ist dort zulässig oder nicht? Also spannende Fragen, die wir noch nicht geklärt haben, die die nächsten Jahre kommen werden.

\Fabian Denken Sie, von dem organisatorischen Aspekt abgesehen, dass es schon möglich wäre, diesen Bereich auf das Internet umzustellen?

\Andre Also es gibt einen Landkreis aus Thüringen, die als Pilot-Landkreis arbeiten damit und die einen gewissen Teil der einfachen Vorgänge - Einfamilienhäuser, etwas größere Wohnhäuser, voll digital bearbeiten. Und ich habe mit Kollegen da gesprochen, die sagen, wir hatten anfänglich Schwierigkeiten und wir mussten noch ein bisschen geschoben werden, um das alles zu machen. Heute sind wir sehr zufrieden damit. Es geht rascher, es geht mit weniger Aufwand. Wir haben die Technik soweit im Griff. Und dann wurde mir aber gesagt, bedauerlich ist, dass wir nicht mehr so viel persönlichen Kontakt haben. Die Architekten kommen gar nicht mehr, sondern sie schicken nur noch etwas. Früher haben wir mal eine Tasse Kaffee getrunken und über ein Problem geredet. Heute wird nur noch gechattet und das wird eher bedauernd gesehen, aber insgesamt eine positive Bilanz, Zwischenbilanz gezogen. Es wird also noch viel weiter gehen.

\Fabian Außerdem haben wir bemerkt, dass sie auf Ihrer Webseite Stellenngebote online gestellt haben, zurzeit eins. Welche Erfahrungen haben Sie diesbezüglich gemacht?

\Andre Nun ja, der Markt auf unserem Spezialgebiet ist leer. Also die guten Leute sind alle angestellt oder selbstständig. Und es gibt nicht die Resonanz, die wir uns wünschen würden. Also wir haben dort immer mal Anfragen, vor allen Dingen wenn die Studienabgänge zu Ende sind. Sind Leute ihren Master oder Bachelor haben, dann suchen die ja frische Anstellungen. Jetzt werden wir wieder jemanden einstellen im Herbst, der in Magdeburg studiert hat, aber das jetzt wie in der Massenpersonalvermittlung Sekretärin gesucht werden oder Sachbearbeiter, das haben wir dort ja nicht. Diese Stellen, die wir auch ab und zu brauchen, die werden wir also auf anderem Wege finden. Und dort ist es so, dass wir quasi immer wieder suchen, weil wir gut zu tun haben und Aufträge kommen. Aber für diese Spezialgebiete, für diplomierte oder mit Master ausgestatteten Fachleute ist der Markt sehr dünn. Da braucht man eigentlich eher ein Headhunter als eine Online-Seite.

\Fabian Hat Ihre Webseite noch andere Funktionen, die wir noch nicht angesprochen haben?

\Andre Ja, das, was wir, was ich am Anfang sagte, was etwas hinkt das würden wir gerne in nächster Zeit verbessern. Einfach das, was wir geleistet haben, auch mal in Bild oder Zeichnung oder Kommentar zu präsentieren, weil das auch für den Mitarbeiter so ein Stück selbstwerterhöhende Reflexion ist. Also die sehen

\end{description}
