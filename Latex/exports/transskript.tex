\addsec{Experteninterview}
Interviewpartner: \hfill Prof. Dr.-Ing. André Spindler\\
Interviewdatum: \hfill 19. 09. 2020\\
Ort des Interviews: \hfill Erfurt\\
\vspace{0.4cm}
00:00:00 - 00:40:19\\
\vspace{0.4cm}

\footnotesize
\setspeaker{Andre}[Prof. Spindler]
\setspeaker{Fabian}[Fabian Beez]
\setspeaker{Toni}[Toni Hausdörfer]
\addtolength{\transcriptlen}{1em}

\begin{description}

\Fabian Wir führen jetzt ein Interview mit André Spindler. Er ist selbständiger Architekt und leitet sein eigenes Architekturbüro mit Spezialisierung auf Brandschutz. Möchten Sie hierzu noch etwas ergänzen? \#00:00:26\#

\Andre Ja, vielleicht zu meiner Person. Ich bin über die 60 weg und habe eine fast 40 jährige Berufserfahrung und hab als Bauingenieur studiert, bin jetzt Architekt seit vielen Jahren und habe mich in das Nischendasein eines Brandschutzfachingenieurs hineingearbeitet. Prüfe dort auch das, was andere machen im Auftrag der Landratsämter und berate sehr viel auf dem Gebiet und hab daneben noch eine Professur an der Fachhochschule Erfurt für Baukonstruktion. \#00:01:05\#

\Fabian Denken Sie, dass die Spezialisierung auf ein Gebiet im Bereich Architektur effektiver ist als sein Wissen breit zu fächern? \#00:01:17\#

\Andre Oh, das ist eine ganz interessante Frage, denn ich kann mich noch gut erinnern an meinen Professor, der gesagt hat "Spindler, gehen Sie erst mal in die Breite. Gucken Sie auf allen Gebieten, was es da gibt. Und wenn Sie das spüren in sich nach einigen Jahren, dann gehen Sie in die Tiefe. Nehmen Sie also ein Spezialgebiet und werden Sie auf dem Gut". Und ob ich das nun bewusst gemacht habe, weiß ich nicht. Aber zumindest ist es bei mir so passiert, dass ich auf eine Vielzahl ganz unterschiedlicher Fachgebiete zurückgreifen kann, in meiner Spezialtätigkeit, und da sich dieses System, überall ein Stück zu wissen, aber auf einem Gebiet sehr viel zu wissen für mich als sehr gut herausgestellt hat. \#00:02:01\#

\Fabian Wie sieht Ihr durchschnittlicher Arbeitsalltag aus? \#00:02:11\#

\Andre Das ist eine ganz schlimme Frage, weil in der Arbeitsbelastung ist man allgemein als Selbstständiger sehr hoch belastet. Und bei mir durch diese mehreren Standbeine und Tätigkeiten ist es deutlich höher als im Durchschnitt. Ich habe also zwei Tage die Woche, in denen ich an der Hochschule tätig bin, mich mit Vorlesungen, Seminaren beschäftige, studentische Fragen beantworte und mehrere Tage die Woche, wo ich mich um mein Büro kümmere und auf Baustellen unterwegs bin. Und oftmals ist es eben das Wochenende, wo ich mich dann um Abrechnungen kümmere, um spezielle Aufgaben, die man als Büroleiter von zehn Beschäftigten hat, am Monatsende Abrechnungen zu machen und ähnlichem. Also der Arbeitstag ist mehr als voll - etwas, was ich am Anfang dieser ganzen Tätigkeiten auch nicht so geahnt habe, dass es so viel Arbeit ist. Ich kann dir aber nicht abwählen, die ist nun mal da. Sie macht mir aber auch Spaß und ich habe zum Glück eine Familie, die das akzeptiert, dass ich lange und abends im Büro bin und auch mal am Samstag mich dort beschäftige. \#00:03:25\#

\Fabian Wir haben ja schon voerherein herausgefunden, dass Sie eine eigene Website haben. Dieses Interview führen wir zu einem großen Teil unter dem Aspekt Onlinehandel bzw. -präsentation - deshalb spielt ihre Wesite aus unserer Sicht eine bedeutende Rolle. Deshalb haben wir auch ein paar Fragen in diese Richtung. Erst einmal ganz allgemein: Seit wann haben Sie ihre Website und seit wann ist es in Ihrer Branche üblich, sich online zu präsentieren? \#00:04:12\#

\Andre In der Branche ist es eigentlich seit Anfang an üblich – also seit dem es dieses Medium gibt, weil die Architekten natürlich im besonderen Maße visuell werben, also mit ihrem Werk zeigen, was sie gebaut haben, sie zeigen vielleicht auch die einzelnen Phasen der Planung und der Fertigstellung. Mit Fotos, mit Videos, mit Drohnenüberflügen und ähnlichem. Das wirbt quasi für die Qualität des Architekten und zeigt auch dem potentiellem Bauherrn, was schon in welcher Qualität geleistet wurde. Das Internet ist also ein ganz wichtiges Medium. Ich weiß auch, dass vorallem die großen Büros viel Wert darauf legen, aktuelle Projekte vorzustellen und sich damit interessant zu machen. \#00:05:01\#

\Toni Würden Sie demnach sagen, dass die Onlinepräsentation wesentlich wichtiger ist als die Eigenpräsentation durch Printmedien? \#00:05:12\#

\Andre Ja, das denke ich schon. Das hat extrem zugenommen, denn bevor es Onlinepräsentationen gab, war eigentlich nur die Möglichkeit der Veröffentlichung in Zeitschriften an den potentiellen Bauherrn heranzutreten oder über das eigentlich gebaute Werk, also jemand der ein Haus gebaut hat, der hatte Bekannte die auch ein Haus bauen wollten. Da hat man den gefragt, wer war denn da Architekt? Wer hat denn für dich geplant? Und wenn man zufrieden war mit dem Architekten, dann hat man den empfohlen. Und das ist schwierig, mit einer solchen Art und Weise der Werbung, zu überleben.

Außerdem werden auch Wettbewerbe ausgeschrieben, außerdem auch online, mittlerweile in fast allen Fällen. Man beteiligt sich an einem Wettbewerb, und wenn man da gut ist, bekommt man einerseits für die Bearbeitung ein wenig Geld, aber wenn man den Auftrag tatsächlich bekommt, dann ist das auch sehr wirtschaftlich. Aber es kann eben bei 50 Teilnehmern nur einer gewinnen – also ist die Chance nicht sehr hoch. \#00:06:20\#

\Fabian Sie hatten ja auch vorhin erwähnt, dass Sie Bauprojekte online stellen, um diese zu präsentieren. Ich habe jetzt auf Ihrer Webseite leider keins gefunden. Wo würde das denn üblich hochgeladen werden? \#00:06:37\#

\Andre Ja, wir sind kein gutes Beispiel für eine gepflegte Website. Ich will das jetzt auch nicht groß entschuldigen, weil für uns diese Dinge, die ich eben gerade über Architekten gesagt habe, nicht ganz zutreffen. Wir bilden ja so eine Teilleistung des Architekten oder Ingenieurdaseins ab - Wir beraten, wir haben spezielle Aufgaben bei der Planung am Bau und die lassen sich nicht so gut darstellen wie das fertige Werk. Wir würden, das ist immer noch so in der Planung und teilweise in der Vorbereitung spezielle Planung vorstellen. Natürlich, die fertigen Häuser auch. Aber wir werben ja nicht mit einer schönen Fassade wie der Architekt, sondern wir werben mit einer Fachplanung, die man möglichst gar nicht sehen soll, sondern die dann wirkt, wenn es brennt und die vorher gar nicht in Erscheinung treten soll. Sie alle kennen sicherlich die die Feuerlöscher, die grünen Männchen über den Türen und solche Dinge, die man eben wahrnimmt. Wir machen natürlich viel mehr in Konzeption, die sich nicht so leicht visualisieren lassen. Also das ist die fachliche Ausrede für das, was wir nicht leisten. Und eine andere Ausrede ist einfach, dass wir ganz wenig Zeit haben und dass wir zwar viel leisten auf diesem Gebiet, aber uns nicht gut verkaufen. Und auch das hat wieder den Grund, dass wir eher durch Empfehlungen - also gar nicht so sehr über online, sondern durch unsere Leistung seit Jahren immer wieder Kundschaft bekommen, vor allen Dingen Architekten, spezielle Bauherrengruppen, auch größere institutionelle Bauherren, die sagen, Ihr habt das gut gemacht, seit Jahren, wir würden euch gern wieder mit fürs nächste Projekt nehmen. Die schauen eben nicht ins Netz, weil die sagen naja, die haben jetzt ein neues Bildchen. Aber eigentlich wollen wir ja die Leistung haben. \#00:08:41\#

\Fabian Also würden Sie sagen, dass mehr Kunden analog als online auf Sie stoßen? \#00:08:49\#

\Andre In unserem Fall ja. Wir kriegen natürlich übers Netz Projektphasen zugesandt. Also Vorplanung, Ideen. Das ist natürlich heute üblich über online und nicht als Papierstapel. Und wir geben Angebote natürlich übers Internet ab. Alles digital. Wir bearbeiten. Jeder hat also mindestens einen Computer an seinem Arbeitsplatz, mehrere Bildschirme. Wir arbeiten nur digital. Aber das Herantreten, das erfolgt, wenn man so will, Analog. \#00:09:24\#

\Toni Gibt es einen grossen Unterschied im Vergleich, als Sie eine Webseite gemacht haben bezüglich der Nachfrage vorher und danach? \#00:09:39\#

\Andre Es gibt einen Unterschied, aber der ist nicht signifikant für die wirtschaftliche Ausbeute, um das mal so zu sagen. Also es gibt eine Reihe von Anfragen über die Website, weil wir dort auch ein Kontaktfeld haben und wir schätzen aber ein, dass das eher zufällige Dinge sind. Wenn das Wort Architekt googelt findet man uns natürlich auch und hat aber, wenn man näher hinschaut, dann eigentlich nicht den Architekten, der in Einfamilienhäusern plant und baut oder eine kleine Sanierung macht. Da haben wir auch Anfragen, die sich speziell auf meine Person beziehen, weil ich eben eine Menge Erfahrung auf dem Gebiet habe, im Denkmalbereich und in Altbausanierung. Aber die meisten speziellen Anfragen zum Brandschutz, zur Planung, die kommen also unabhängig von der Website. \#00:10:42\#

\Toni Noch eine Frage wäre beispielsweise, wie der Online-Aspekt, also diese Webseiten und die mediale Repräsentation, wie die sich auswirken können, vielleicht auf ihr berufliches Umfeld und auch auf die konkurrierenden Firmen. \#00:11:05\#

\Andre Also wir nutzen unsere Homepage nicht nur für Werbung oder Verkauf, in Anführungsstrichen. Wir verkaufen ja Wissen und das können wir nur sagen, dass dies der Fall ist. Aber wir verkaufen es nicht online wie Online-Handel. Wo wir gute Erfahrungen gemacht haben, ist die Nutzung dieser Online-Möglichkeit, bei der Information von nachgelagerten Personen. Das will ich kurz erläutern: Wir haben also ein Teil unserer Website mit Formblättern, mit Beantwortung von Fragen dieses "FAQ", z.B. was viele Firmen haben gefüllt. Das heißt, ich verlange in meiner Prüftätigkeit am Abschluss eines Bauvorhabens bestimmte Unterlagen, Nachweise, Unterschriften, Fotos und ähnliches. Das haben wir alles auf unserer Homepage dargestellt. In welcher Form wir das wollen; was, was wir unter bestimmten Begriffen verstehen, unter bestimmten Nachweisen fordern müssen. Und der Bauleiter, der Architekt, auch der Bauherr kann dort nachschauen und kann sich diese Informationen online herunterziehen und ist damit natürlich nicht mehr eine Belastung für uns. Wir müssen nicht alles am Telefon erklären oder ihn einladen, dass wir ihn schulen. Sondern er kann dieses Medium nutzen und sich dort die Formblätter zum Ausfüllen herunterladen, damit arbeiten. Das halte ich für einen ganz großen Gewinn. Das wird auch reflektiert von den Baustellen, dass man sagt ja, das ist gut. Wir können doch abends um 10 uns etwas runterziehen, das durchlesen und morgen um 8 verlangen wir diese Dinge auf der Baustelle. Zu den Mitbewerbern war so ein bischen eine Frage. Da muss ich sagen, auf dem Niveau, auf dem wir arbeiten, haben das viele. Ich denke, die Mehrzahl der Büros unterstützen damit die ihnen nachgelagerten Baustellen. \#00:13:08\#

\Fabian Noch eine Frage, die Sie eigentlich schon beantwortet haben - Spielt die Onlinepräsenz in ihrer Branche eine bedeutende Rolle? \#00:13:20\# %(doppelt?)

\Andre Bei Architekten ja, bei unserem Spezialgebiet eher weniger. \#00:13:27\#

\Fabian Welche Rolle spielt ihre Website in dem Konsumverlauf – Werden Kunden über die Seite auf Sie Aufmerksam oder hat sie eher eine Informationsfunktion? \#00:13:40\#

\Andre Ich denke, es ist mehr die Informationsfunktion. Natürlich gibt es auch zufällige Kunden, die das Wort Architektur oder Brandschutz eingeben, auf uns stoßen. [...] Unsere Webite enthält die technischen Informationen und die Gesichter der Mitarbeiter, was ich übrigens auch für wichtig halte – mit wem man vielleicht ein Jahr zu tun haben wird, oder wer immer wieder die Fragen beantwortet oder die Unterlagen anfordert, dass man da mal ein Gesicht sieht – diese menschliche Basis. Dass man unsere Leistung mehr konsumiert, dass kann man nicht erwarten. Die Leute kommen ja nicht zu uns, um eine Handtasche zu kaufen, sondern um eine spezielle Dienstleistung von uns erledigen zu lasssen. Dort sind wir sicherlich auch immer wieder im Wettstreit mit Mitbewerbern, die das ählich anbieten, vielleicht auch einmal etwas kostengünstiger, oder die andere Vorteile haben, weil sie etwa in der Nähe des Vorhabens ihren Bürositz haben und damit schneller verfügbar sind, oder ähnliche Dinge. \#00:14:51\#

\Fabian Inwiefern ist die Menschliche Komponente im Onlinebereich, die Sie soeben angesprochen haben, wichtig für Sie? \#00:14:58\#

\Andre Ich habe ja schon erwähnt, dass die Mitarbeiter dort vorgestellt werden, natürlich mit deren Einverständnis - Wir filmen jetzt niemand heimlich und stellen ihn ins Netz. Wir hatten da eine Fotografin, die auch darauf geachtet hat, dass wir da gut rüber kommen und uns ordentlich präsentieren. Auf der Website wird ganz kurz eingeschätzt, natürlich neben dem Namen, welche Qualifikationen der Mitarbeiter oder die Mitarbeiterin haben, wofür sie zuständig sind, und das ist auch sehr geschickt gemacht. Man sieht immer das gesamte Team, und wenn man auf den Mitarbeiter mit der Maus geht, wird er farbig und seine Daten erscheinen. Ich halte das für sehr angenehm und das führt schon dazu, dass man ein persöhnlicheres Verhältnis zueinander hat. Ich sehe das auch in meiner Seminartätigkeit, mit der ich in Deutschland unterwegs bin. Die meisten kennen mich, weil man mich irgendwann mal gegoogelt hat und dan weiß, wie ich aussehe. Dann begrüßt man mich ganz freundlich, dann sage ich, ich kenne Sie ja leider gar nicht, weil mir das Gesicht eben nicht eingängig ist, aber manche kennen mich weil ich schon einmal irgendwo bekannt geworden bin, über solche Medien. \#00:16:17\#

\Fabian Wir haben auch auf Ihrer Webseite gesehen, dass Sie Mustervorlagen für Erklärungen zur Verfügung stellen. Sie haben schon gesagt, das ist ein für das Bauwesen an sich. Haben diese doch einen anderen Nutzen? \#00:16:31\#

\Andre Ja, manche Dinge sind ja auch nur, um aufklärend zu wirken. Also was steckt hinter bestimmten Begriffen? Ich sag mal ein Beispiel wenn ein Handwerker eine Feuerschutztür einbaut, dann wollen wir wissen von welcher Firma? Was kann dieses Produkt, diese Tür? Und wir wollen andererseits wissen hat es der Handwerker auch richtig eingebaut, hat er ja die Montageanleitung gelesen und verstanden und auch die richtigen Schrauben, die richtigen Dübel genommen. Und es wird kurz erläutert dort der erfahrene Handwerker wird zur sagen, ja, das kann ich schon, das weiß ich, das muss ich gar nicht lesen. Aber am Ende ist es schon gut, dass wir die gleiche Sprache sprechen und eben quasi auch ein, ich will sie jetzt nicht Weiterbildung nennen, aber zumindest eine Konformität mit dem Wissen und Können der Handwerker erzeugen, sodass wir es am Ende leicht haben, die Leistung zu akzeptieren. \#00:17:28\#

\Toni Wie hat denn die Modernisierung Einfluss darauf genommen, wie Sie beispielsweise Unternehmen wählen, die sie dann beauftragen? \#00:17:43\#

\Andre Ja, wir beauftragen selber nicht, sondern das machen dann die Architekten und die Bauherren, aber ich will ein anderes Beispiel erzählen, wo die Reise hingehen wird. Und da sind wir gerade dabei. Das wird also von Papier weitestgehend in der ganzen Genehmigungsphase, im Umgang mit den Bauämtern und den Prüfingenieuren wie mir, auf Elektronik umgestellt werden. Die nächsten zwei, drei Jahre. Das heißt, wir werden kein Papierbündel mehr bekommen zum Durchsehen und Abstempeln, sondern nur noch Dateien. Und das klingt so einfach. Ist es gar kein Problem Dateien zu versenden. Das machen wir ja seit vielen Jahren. Aber die Frage ist, wie werden die gespeichert? Über 30 oder 50 Jahre, weil es ja Dokumente sind. Wenn also in 20 Jahren ein Kindergarten brennt, wird man mich versuchen aufzusuchen mit meinem Wissen, mit meinen Dokumentationen. Es hat dort richtig gemacht worden? Und das muss man auch in elektronischer Form und auch der Austausch übers Internet sichern und handlebar machen. Da sind viele Fragen noch ungeklärt. Auch die Ämter, auch die zuständigen Ministerien wissen noch nicht genau, wo eben die Reise hingeht. Und es muss für jeden handhabbar sein, auch für den kleinen Planer auf dem Dorf, der vielleicht ein oder zwei Einfamilienhäuser macht. Auch der muss sich diesen elektronischen Gangarten unterwerfen. Und das wird spannend die nächsten Jahre. Ich mache zum Beispiel auf jede Zeichnung einen grünen Stempel und sage ja, die wurde von mir geprüft. Da unterschreibe ich den Stempel, trage eine Nummer ein und das Datum. Und ein Kollege hat mal so aus Spaß gesagt, wir können doch nicht auf dem Bildschirm stempeln. Wie machen wir das dann? Wie machen wir ein solches Dokument als geprüftes Dokument kenntlich? Wie sichern wir diese Prüfung, dass sie niemand fälschen kann? Sie wissen sicherlich besser sogar als ich, wie schnell man Elektronik auch missbrauchen kann. Und in Papierform ist das schwieriger möglich, ist auch möglich, aber in der Elektronik viel leichter. Und wie erfährt zum Beispiel auch der dänische Eisenflechter, der mit einem französischen Fahrer unterwegs ist, auf einer belgischen Baustelle? Welches ist jetzt der letzte Stand der Zeichnung? Was hat Spindler geprüft? Was ist dort zulässig oder nicht? Also spannende Fragen, die wir noch nicht geklärt haben, die die nächsten Jahre kommen werden. \#00:20:23\#

\Fabian Denken Sie, von dem organisatorischen Aspekt abgesehen, dass es schon möglich wäre, diesen Bereich auf das Internet umzustellen? \#00:20:35\#

\Andre Also es gibt einen Landkreis aus Thüringen, die als Pilot-Landkreis arbeiten damit und die einen gewissen Teil der einfachen Vorgänge - Einfamilienhäuser, etwas größere Wohnhäuser, voll digital bearbeiten. Und ich habe mit Kollegen da gesprochen, die sagen, wir hatten anfänglich Schwierigkeiten und wir mussten noch ein bisschen geschoben werden, um das alles zu machen. Heute sind wir sehr zufrieden damit. Es geht rascher, es geht mit weniger Aufwand. Wir haben die Technik soweit im Griff. Und dann wurde mir aber gesagt, bedauerlich ist, dass wir nicht mehr so viel persönlichen Kontakt haben. Die Architekten kommen gar nicht mehr, sondern sie schicken nur noch etwas. Früher haben wir mal eine Tasse Kaffee getrunken und über ein Problem geredet. Heute wird nur noch gechattet und das wird eher bedauernd gesehen, aber insgesamt eine positive Bilanz, Zwischenbilanz gezogen. Es wird also noch viel weiter gehen. \#00:21:35\#

\Fabian Außerdem haben wir bemerkt, dass sie auf Ihrer Webseite Stellenngebote online gestellt haben, zurzeit eins. Welche Erfahrungen haben Sie diesbezüglich gemacht? \#00:21:50\#

\Andre Nun ja, der Markt auf unserem Spezialgebiet ist leer. Also die guten Leute sind alle angestellt oder selbstständig. Und es gibt nicht die Resonanz, die wir uns wünschen würden. Also wir haben dort immer mal Anfragen, vor allen Dingen wenn die Studienabgänge zu Ende sind. Sind Leute ihren Master oder Bachelor haben, dann suchen die ja frische Anstellungen. Jetzt werden wir wieder jemanden einstellen im Herbst, der in Magdeburg studiert hat, aber das jetzt wie in der Massenpersonalvermittlung Sekretärin gesucht werden oder Sachbearbeiter, das haben wir dort ja nicht. Diese Stellen, die wir auch ab und zu brauchen, die werden wir also auf anderem Wege finden. Und dort ist es so, dass wir quasi immer wieder suchen, weil wir gut zu tun haben und Aufträge kommen. Aber für diese Spezialgebiete, für diplomierte oder mit Master ausgestatteten Fachleute ist der Markt sehr dünn. Da braucht man eigentlich eher ein Headhunter als eine Online-Seite. \#00:23:06\#

\Fabian Hat Ihre Webseite noch andere Funktionen, die wir noch nicht angesprochen haben? \#00:23:13\#

\Andre Ja, das, was ich am Anfang sagte, was etwas hinkt, das würden wir gerne in nächster Zeit verbessern. Einfach das, was wir geleistet haben, auch mal in Bild oder Zeichnung oder Kommentar zu präsentieren, weil das auch für den Mitarbeiter so ein Stück selbstwerterhöhende Reflexion ist. Also die sehen dann ja, guck, da haben wir mit gemacht. Und ich merke das bei unseren jüngeren Mitarbeitern, wenn wir machen weiter so durch die Stadt gehen oder radeln, einen Termin haben. Da gucken die schon immer nach den Häusern, die sie bearbeitet haben. Und das ist für sie auch ein stolzes Gefühl, wo wir überall mitgearbeitet haben und das eben ins Netz zu stellen, "Guckt her, das haben wir geleistet" - Da sind wir schwach. Aber das hab ich mir vorgenommen, das unbedingt noch zu machen. \#00:24:04\#

\Fabian Im Blick auf andere Bereiche, wie z.B. die Einstellung als Dozent, welche Rolle spielen die Anbindung ans Internet in diesen, speziell auf Sie bezogen? \#00:24:22\#

\Andre Also da spreche ich jetzt mal nicht von unser Büro Homepage, die hat ja auch einen Hinweis, dass ich dort tätig bin. Aber wir haben natürlich an der Hochschule eine hervorragende Website mit ganz vielen Funktionen, die für meine Tätigkeit und für die Studenten wichtig sind. Ich habe, als ich angefangen habe, sofort mit digitalen Unterlagen begonnen. Also vor 15 Jahren war das schon durchaus üblich, aber mein Vorgänger hat das nicht gemacht und ich hab das bei Null aufgebaut und alle Vorlesungen sind digitalisiert. Alle Skripte können sich die Studenten herunterladen und ergänzen. Während der Vorlesung werden das Seminare. Wir haben so eine Art Rohlinge, die wir den Studenten zur Verfügung stehen, die dann im Seminar sitzen, auch mit Laptops und dort weiter zeichnen. Also während wir das erläutern, auf was es ankommt, dann die Zeichnungen ergänzen, abspeichern und damit auch später mal im Beruf sich erinnern können, was sie da studiert haben, dazu kommennatürlich auch die Bibliothek und alles Mögliche, was wir nutzen, auch im Büro nutzen dürfen, um Normen anzusehen, um uns zu informieren, wie der Stand der Technik ist. Das läuft also alles über das Internet und ist ein hervorragendes Mittel. Auch Dinge, die wir früher eben im Buch nachgeschlagen haben, können wir jetzt eben online oder eben auch über andere Medien ansehen und dort auch z.B. mal ein Bild mit benutzen, um meinen Gedanken zu erläutern, können das raubkopierern. Dafür haben wir Lizenzen - dafür bezahlen wir Geld. Und das ist ein hervorragendes Medium, was gerade von jungen Leuten gerne genutzt wird. \#00:26:11\#

\Fabian Noch in Bezug auf die Corona-Zeiten der letzten paar Monate, währe der Unterricht ohne Online-Konferenzen oder ähnliches analog überhaupt möglich gewesen? \#00:26:26\#

\Andre Nein, also die Hochschulleitung und die gesamten Ministerien haben ja entschieden, dass die Hörsäle zubleiben, um eine Ansteckung der Studenten untereinander zu vermeiden. Auch deshalb, weil im Gegensatz zu einer Schule die Studierenden ja von überall herkommen. Wir haben einen Anteil von über 40 Prozent von Studierenden, die nicht aus Thüringen kommen und davon nochmal auch einen gewissen Ausländeranteil. Und da hat man einfach keine Kontrolle. Und das verstehen wir auch. Deswegen wurde dann entschieden Das Sommersemester, das jetzt zu Ende geht, online durchzuführen. Wir haben spezielle Übertragungsprogramme, die die Studenten auch haben und ich hatte meine Seminare, Übungen, Vorlesungen bis zu den Semesterferien online gemacht, mit anfänglichen Schwierigkeiten, haben alle Studenten ein schnelles Internet oder komme ich da nur ruckelig rüber oder ähnliches auch mit dem vorsichtigen Versuch, auch in den Dialog zu geraten. Mal Fragen zu beantworten und ähnliches. Im Moment sind wir damit aber sehr zufrieden. Wir haben viel dazugelernt. Wir zeigen also auch Zwischenergebnisse an alle Studierenden. Wir diskutieren darüber gemeinsam, sofort. Also nicht auf Band und alle haben etwas davon. Können also lernen aus den Fehlern der anderen oder aus den positiven Ergebnissen der anderen. Und insgesamt bin ich also mit den Leistungen der Studenten zufrieden. Wir haben das Niveau nicht verloren. Das sehe ich an den Prüfungen, die wir aber präsent gemacht haben, also nicht online. Und wir werden das kommende Semester auch wieder online machen. Was fehlt und was von den Studenten noch mal wieder kritisiert wird, ist wirklich der persönliche Kontakt. Auch mal Kontakt im Biergarten, um mal so ungezwungen miteinander zu reden. Wir machen das gerne. Wir haben da ein lockeres Verhältnis zu den Studenten und das fehlt natürlich. \#00:28:36\#

\Fabian Sie haben schon gemeint, dass es Anfangsschwierigkeiten gab. Insbesondere, da Thüringen ein sehr intensives Internetz hat. Und wenn ja, wie sah diese Anfangsschwierigkeiten sonst aus? \#00:28:54\#

\Andre Also wir haben am Anfang auch keine guten Programme gehabt. Wir haben also mehrere Programme ausprobiert und unser Hochschul-Rechenzentrum hat dann eines ausgewählt, was gut ist, was viele Möglichkeiten hat. Auch in der Tiefe zum Beispiel, Gruppen zu bilden, Aufgaben zu stellen, die dann eingelagert werden, die Lösung eingelagert werden, mit denen man dann individuell beraten kann. Sind die Lösungen gut, musst du dort noch weiter arbeiten und ähnliches. Das waren also Dinge, wo halt auch niemand darauf eingestellt war in den Fachkreisen. Wie kann man also eine Vorlesung mit 120 oder 80 Studierenden machen, dass die alle was davon haben? Dann muss ich aber auch sagen, dass wir menschlich darauf nicht eingestellt waren, dass man natürlich weiß, ich gehe eine Vorlesung, ich hab auch ein Stück Kreide dabei und kann das, was ich als mit dem Computer an die Wand werfe, mit dem Beamer auch ergänzen. Nochmal durch eine kleine Skizze. Das gab es dann nicht. Ein Kollege hat dann sein Handy auf ein Gestell gezwackt, sodass es wie eine kleine Kamera ihn beobachtet hat, wenn er eine Skizze dort fertigt. Und dann auch die Frage - Wo arbeiten wir eigentlich? Ich bin anfangs an die Hochschule gefahren, weil wir dort ein Giga-Netz haben und konnte da sehr gut meine Vorlesung machen. Und als mein Büro dann auch 115-MB-Netz hatte, dann bin ich wieder ins Büro gegangen, weil ich da alle meine Unterlagen habe und habe von dort aus Vorlesungen gehalten. Also Dinge, die ich auch niemandem vorwerfen möchte. Das sind einfach so Kinderkrankheiten. Und wenn man von heute auf morgen sich umstellen muss, dann ist das erst mal ganz normal. Studenten haben da auch locker reagiert, wie auch auf Probleme, die man uns angezeigt hat. Wir hören euch nicht. Dann muss man dafür Verständnis haben. Mittlerweile, so dass ich oder meine Vorlesungen mitschneiden und ins Netz stelle, sodass auch ein Student, der nicht kann, der Laborversuch gerade macht und nicht zu dieser Vorlesung kommen kann, sich das später ansehen kann. \#00:31:08\#

\Fabian Um noch einmal auf den Bereich Architektur zurückzukommen - Wie denken Sie, wird sich der Online-Aspekt in Zukunft entwickeln? Stimmen Sie der klassischen Annahme, dass es wichtiger wird, in Bezug auf ihre Branche zu? \#00:31:22\#

\Andre Ja, also wir sind jetzt in dieser BIM-Phase. Habt ihr schon gehört? BIM? [...] Das ist natürlich ein Fachgebiet aus der Baubranche - Building Information Modelling. Also wir haben jetzt seit Jahren mit Computern gezeichnet, in 2D und in 3D und mit dem BIM Systemen, die jetzt gerade bei den Großprojekten angewendet werden sollen und müssen, wird eben der Aspekt der konfliktlosen Planungen in den Mittelpunkt gestellt. Wenn wir also als Architekten einen Entwurf gemacht haben, der Statiker dann die Dicke der Stützen bestimmt und der Haustechniker eine Leitung durchs Haus gelegt, hat man gemerkt, ja, da wo jetzt eine Leitung liegt, müsste eigentlich Bewährung einer Stütze sein. Und wenn man das erst so bei der End-Planung merkt, ist das natürlich schlimm. Da muss man alles wieder von vorne beginnen. Und so werden quasi über solche BIM-Schnittstellen, die natürlich online verlaufen müssen, solche Konflikte sehr viel früher erkannt. Und das Zweite ist, dass wir den Bauteilen einer Wand, einer Tür, einem Teppichboden, einem Heizkörper Eigenschaften zuordnen. Diese Eigenschaften bleiben dann an dem Bauteil ein Leben lang haften. Wenn also ein Facility Manager in 20 Jahren sagt, oh, der Heizkörper tropft, dann kann er feststellen, was ist das für ein Fabrikat oder wer hat das eingebaut? Werden die überhaupt noch hergestellt? Was hat er für eine Heizleistung? Wo kriege ich den her? Und wir haben das über Jahrzehnte erlebt. Man macht eine Dokumentation zu Ende und dann landet ihr am Dachboden und niemand weiß, wie gehe ich damit um? Wir fangen eigentlich immer wieder von vorne an und das will man mit solchen Vernetzungen, elektronischen Vernetzungen vermeiden. Und da sind auch viele Großprojekte mittlerweile so als Pilotprojekte auf dem Weg dahin. Und wir werden da nicht drum herum kommen. Wir werden uns da anschließen, weil unsere Dinge - eine Feuerschutz Tür natürlich genauso mit eingebunden werden muss wie die Statik oder die Gestaltung des Gebäudes. \#00:33:37\#

\Fabian Denken Sie, dass die Online-Präsentation in Zukunft auch auf weitere Bereiche wie zum Beispiel 3D-Animationen auf Ihrer Website oder ähnliches sich erweitern wird? Und wenn ja, welche Bereiche? \#00:33:52\#

\Andre Da würde ich jetzt mal sagen, das überlasse ich meinen jüngeren Mitarbeitern, die das vielleicht mal weiterführenden, das Büro. Ich denke, dass wir mit 3D, mit dem wir ja arbeiten, also viele Projekte bekommen wir als 3D-Dateien und arbeiten dann auch damit, dass die eine Rolle des Informationsaustausches mit anderen an Bedeutung gewinnen werden. Also wir sind da soweit - wir können das eigentlich seit Jahren, geben das die Zeichenprogramme schon her, nicht so sehr jetzt im Sinne der Werbung oder sowas, sondern einfach als Austauschmedium für die tägliche Arbeit. Und da bin ich ganz optimistisch. Aber ich muss zugeben, ich habe ja am Anfang mein Alter angedeutet, dass es vielleicht doch besser ist, wenn die nächste Generation - ich habe insgesamt ein sehr junges Büro, viele noch unter 30. Dass die sich dem annehmen und sich dann wirklich mit ihrem frischen Wissen und auch mit dem Wissen, dass sie das brauchen werden, dort engagieren. \#00:34:55\#

\Toni Wie denken Sie, wie nützlich eine Erweiterung Ihrer Website auf andere Sprachen wäre? Das heißt, eine Webseite kreieren, wo auch anderen Netzen ihr Unternehmen gezeigt werden könnte? \#00:35:21\#

\Andre Also andere Sprachen, also Englisch oder so etwas meinen Sie jetzt? \#00:35:27\#

\Toni Bespielsweise. \#00:35:29\#

\Andre Also da haben wir keine Erfahrung, muss ich zugeben. Der deutsche Markt ist groß genug im Moment, um vielen, die so ingenieurmäßig oder als Architekten, tätig sind, Lohn und Brot zu bieten. Ich weiß aber, dass eine ganze Reihe der größeren Büros eben auch versuchen, international tätig zu sein und sich dort an Wettbewerben beteiligen und ähnlichem. Die haben einfach einen Button, drückt man drauf, ist die ganze Website in Englisch und das ist für die selbstverständlich. Wir haben das nicht, muss ich zugeben. Im Moment sehe ich da auch keinen unmittelbaren Bedarf. Obwohl ich auch immer mal auch im Ausland bin. Mit meiner Hochschul-Tätigkeit, wir werden da gut ausgerüstet. Also wir könnten das sicherlich. Aber wir brauchen es momentan nicht, weil wir eher regional, also in Thüringen, auch in den Nachbarländern tätig sind, und da wird immer noch Deutsch gesprochen. \#00:36:25\#

\Toni Was denken Sie, würde sich eventuell eine Einbindung einer Art Auftragsform in Ihrer Webseite sich positiv oder negativ auswirken, würden Sie das auch machen? \#00:36:42\#

\Andre Auftragsform oder Auftragsformular oder was meinen Sie? \#00:36:46\#

\Toni Praktisch eine Form, die man online ausfüllen kann, damit Sie dann direkt eine Nachricht darauf bekommen? \#00:36:57\#

\Andre Das haben wir. Wir haben also Kontaktformular, mit denen man also seine Adresse oder eine Frage abgeben kann. Aber das war vielleicht nicht der Inhalt. \#00:37:07\#

\Fabian Ich denke, dass er meint, dass ein Auftrag direkt an Sie gesendet wird. \#00:37:12\#

\Toni Ja. \#00:37:13\#

\Andre Nun ja, wir kriegen natürlich über über das Internet Aufträge. Das hat aber jetzt mit der Homepage nichts zu tun. Also wir geben ja Angebote ab und auch teilweise ohne Angebote werden wir beauftragt über öffentliche Stellen und das geht zu einem gewissen Teil online. Aber das ist nur das Transportmedium, und das was Sie sicherlich meinen ist, dass man, vielleicht beim Handel, dass man sagt, da gibt's ein Angebot und das nehme ich jetzt als Kunde an. Sowas geht bei uns nicht. Und zwar nicht nur bei uns nicht, sondern ich glaube, in der ganzen Branche macht man das nicht, weil wir ja nicht ein einzelnes Produkt, ein Paar Schuhe oder ein Fahrrad anbieten, verkaufen, sondern eine recht komplexe Tätigkeit. Und wir brauchen, um überhaupt einen Preis zu finden, dem man dann annehmen könnte, bräuchte man eine ganze Reihe Informationen. Ich kann mir so etwas vorstellen, dass man sagt, wir würden z.B. Flucht und Rettungspläne machen. Habt ihr schon gesehen, wo an der Wand hängen muss? Wo man vorbei geht, wenn man ein Gebäude verlässt, und da könnte man sagen ein Stück kostet 80 Euro. Und das birgt das Risiko, dass wir gar nicht wissen, wie kompliziert ist das Gebäude. Und dann haben wir einen viel zu geringen Preis dort hineingesetzt. Oder wir müssen den Preis hochtreiben, um alle Unwägbarkeiten aufzufangen, dass der Bauherr sagt, das ist ja viel zu teuer. Deswegen, denke ich, wird es in vielen Dingen so sein, dass man erst einmal in den Austausch geraten muss. Also erst einmal sich über den Gegenstand unterhält und dann natürlich auch einen Preis abgibt. Und dann kann man dann online auch bestätigen. \#00:39:08\#

\Toni Als nächstes würde ich Sie einfach nach Ihrer Meinung zu Online-Marktplätz als auch Online-Bezahldiensten fragen. \#00:39:18\#

\Andre Also die Marktplätze werden von uns wahrgenommen und Bezahldienste auch. Das macht also mein Mitarbeiterstab. Wir haben also pfiffige Leute, die gucken, wenn wir spezielle Dinge brauchen in der Elektronik. Jetzt hat ein Kollege sich beispielsweise einen sehr stabilen Laptop bestellt, für die Baustelle, dass der auch mal runterfallen kann; Schutzausrüstung für bestimmte Zwecke, und da gucken die nach und dann bezahlen wir das über verschiedene Medien und haben da bis jetzt immer gute Erfahrungen gemacht, dass das klappt. \#00:40:01\#

\Fabian Okay, das waren unsere Fragen. Wir bedanken uns herzlich für das Interview. \#00:40:11\#

\Andre Ja, gerne. Alles Gute wünsche ich euch. \#00:40:16\#

[...]

\end{description}

\normalsize
